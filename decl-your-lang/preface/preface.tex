\chapter*{Preface}

This is a book about declarative language definition with the Spoofax Language
Workbench.
The aim of Spoofax is to separate the various concerns of language definition
and implementation and provide high-level declarative meta-languages for each
concern.
These meta-languages are declarative in the sense that they abstract from the
\emph{how} of language implementation and focus on the \emph{what} of language
design.
For example, ``what is the syntax of my language?'', instead of ``how do I
implement a parser for my language?''.
Thus, a language designer should not be distracted by language implementation
details. 

\paragraph{Resources}

This book is a response to the demand for better documentation for Spoofax. This
demand suggests a sparsity of material.
However, there is rather a lot of material available online. The
\href{http://metaborg.org}{metaborg.org} website provides information about
installation and documentation about the individual meta-languages. 
Furthermore, there is a large number of research papers about Spoofax and its
meta-languages; see for example the online
\href{http://researchr.org/bibliography/metaborg-book/publications}{bibliography}
for this book.
All that material can be rather overwhelming for new users. 
It may not be quite clear where to start.
Furthermore, since Spoofax and its predecessor Stratego/XT have been
under development for more than 15 years, the (syntactic) details of
papers may be out-of-date.
Hence, the goal of this book is to provide a single coherent guide to the basic
concepts for new users of the workbench.
This book draws on the available material and provides references to papers and
web pages for further reading.

If you are an academic reader, note that this book is \emph{not a primary
source}. All material in this book is based on previously published research. By
all means do cite this book, but that does not absolve you from doing
your homework and citing the proper primary sources. The \emph{Further Reading}
sections in this book should help you in identifying those sources. And the
online bibliography referred to above provides you with ready made
\href{http://researchr.org/downloadbibtex/bibliography/metaborg-book}{bibtex
entries}.

\paragraph{In Progress}

This is a book in progress, just as Spoofax itself is very much in progress.
(Although surely Spoofax is much further along than this book.) Spoofax is the
product of an ongoing research project that investigates the nature of software
language definition. We are continuously experimenting with better abstractions
for various concerns.
As a result, new features are being added to Spoofax and we are also always
working on the guts of the workbench.
Therefore, working with Spoofax can be a somewhat rocky experience, especially
if you are using the latest features, which we are prone to advertise.

Indeed, while I start writing this book in June 2015 using the nightly-build
descendant of Spoofax 1.4, Spoofax is about to enter a new era.
Within the coming months, we are planning to switch to a new version of Spoofax
based on a complete overhaul of the internal architecture of the workbench.
We are replacing IMP, the framework that provided the binding of Spoofax
meta-languages to Eclipse, with Spoofax Core, a framework that defines IDE
services independently from Eclipse. This will allow us to target other IDE
containers such as NetBeans and IntelliJ (modulo engineering work) as well as
support robust command-line implementations of languages defined in Spoofax. So
there is a risk that some of the text of this book and example project that
comes with it, will have to be rewritten some.

\paragraph{GitHub}

This book and the accompanying Spoofax projects are on \BookOnGithub{github}. 
I will consider pull requests with minor or major contributions to the book
and accompanying projects.

-- Eelco Visser

June 15, 2015


