\newpage
\chapter{Declarative Language Definition}

This chapter examines the philosophical background of language workbenches in
general, and Spoofax in particular.
If you just want to get on with it and start building a
language definition with Spoofax, just proceed to \Chapter{getting-spoofax}.

\section{Software Languages}

programming languages

expressing computation

development from directly programming machine code to high-level abstractions

not all languages used in software development can be characterized as
programming languages

For example, HTML is clearly a language 

data formats

domain-specific language

the term `software language' captures this broader class of languages 

Considering all these languages is useful since they share quite some
commonality

All software languages are characterized by a formal syntax. That is, the
well-formed sentences of the languages can be formalized using a formal
system. This can be a (context-free) grammar, but may also be a recognition
algorithm.

Furthermore, the structure assigned to well-formed sentences is assigned a
meaning. The semantics of the language

\section{Understanding Software through Linguistic Abstraction}

pragmatic goals

modeling

capturing domain understanding

\section{}

internal, external, embedded languages

leaky abstractions

\section{Language Workbenches}

what is a language workbench

integrated tool for 

\begin{itemize}
  \item parsing
  \item abstract syntax tree representation
  \item static analysis
  \item code generation (compilation)
  \item interpretation
  \item editor services
\end{itemize}

while the functionality of a language workbench sounds pretty standard, there is
a range of different approaches 

different language implementation philosophies

long tradition of tools to assist language developers

\begin{description}
\item[LEX/YACC] compiler-compiler
\item[Centaur]
\item[ASF+SDF MetaEnvironment]
\item[JastAdd]
\item[Xtext]
\item[MPS]
\item[Spoofax]
\item[Rascal]
\item[SugarJ] 
\end{description}


\section{Declarative Language Definition}

declarative meta-languages

capture our understanding of the domain of language definition

separation of concerns

syntax, name binding, type constraints, dynamic semantics, source-to-source
transformations


tension between declarative expression and programmability

