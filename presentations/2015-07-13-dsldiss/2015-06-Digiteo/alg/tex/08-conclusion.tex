\section{Conclusion and Future Work}\label{conclusion}

We have introduced a generic, language-independent framework for describing name
binding in programming languages.
Its theoretical basis is the notion of a scope graph, which abstracts away from
syntax, together with a calculus for deriving resolution paths in the graph. 
Scope graphs are expressive enough to describe a wide range of binding patterns
found in real languages, in particular those involving modules or classes.  We
have presented a practical resolution algorithm, which is provably correct with
respect to the resolution calculus. We can use the framework to define generic
notions of \a-equivalence and renaming.

As future work, we plan to explore and extend the theory of scope graphs, in
particular to find ways to rule out anomalous examples and to give precise
characterizations of variable capture and substitution.
On the practical side, we will use our formalism to give a precise semantics to
the NaBL DSL, and verify (using proof and/or testing) that the current NaBL
implementation conforms to this semantics.

Our broader vision is that of a complete language designer's workbench that
includes NaBL as the domain-specific language for name binding specification and
also includes languages for type systems and dynamic semantics specifications.
In this setting, we also plan to study the interaction of name resolution and
types, including issues of dependent types and name disambiguation based on
types. Eventually we aim to derive a complete mechanized meta-theory for the
languages defined in this workbench and to prove the correspondence between static
name binding and name binding in dynamics semantics as outlined
in~\cite{VisserOnward14}.

