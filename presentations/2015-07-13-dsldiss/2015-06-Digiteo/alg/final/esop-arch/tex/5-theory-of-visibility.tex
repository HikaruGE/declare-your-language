\section{A Theory of Visibility}

naming conflict: multiple resolution paths for same identifier

some naming conflicts are resolved by policies for visibility 

thesis: shadowing / visibility is independent / orthogonal to resolution paths

systematic analysis of the design space of shadowing as expressed in terms of path well-formedness and specificity ordering
\begin{itemize}
 \item  model different kinds of imports through flags on imports
\item transitive vs non-transitive vs non-transitive+export
\item include vs import vs prefer-imported: maybe use a precedence value in the imports
\item imports with inherited lexical scope (handled by well-formedness)
\end{itemize}

Add flags to imports to describe the different policies, should be handled by well-formedness of the resolution path

\newcommand{\flag}[1]{\overline{#1}}

\paragraph{Import Policies}
To reflect the different imports policies, we add boolean flags to the imports,
when no flag is specified the default behavior is used.
Assume some \emph{import B} is declared in a module {\it A},
\begin{itemize}
 \item If the Import/Export flag $e$ of this import is $0$ then the definition in {\it B} are visible in $A$ and if it is $1$ the definitions
  in {\it B} are also visible from every scope importing \emph{A}.
 \item If the transitivity flag $t$ is $0$ then the resolution can only use the definitions or exports of the imported module else
  the resolution can use all the import clauses in the imported scopes
 \item If the parent (recursive) flag $r$ is $0$ then the resolution process can only continue through possible imports else it 
  can also look into the parent scope
\end{itemize}
The flag of an import is thus a triple of booleans and denoted $\flag{etr}$, for example a non-transitive export that do not allow recursive call
to parent has flag $\flag{100}$.


% The precedence flag $pr^{F}$ can be either set to:
%   \begin{itemize}
%    \item Usual import $u$ whose precedence is lower than the definition, thus a resolution using directly a definition is preferred
%     over an usual import
%    \item Include import $i$ whose precedence is the same than a definition
%    \item Pref import $p$ whose precedence is higher than a definition, such an import is preferred over a direct definition
%   \end{itemize}

??? what if we want priority of import over parents ? for example import with same precedence as parent, parent preferred over imports ??? 
Back to the idea of giving a precedence to imports in algorithm: try to resolve with precedence n, if you find something then done else try with
prec (n+1), parent and definition are only one of these way.

Automata in Figure \ref{fig:algoterm} for the correctness of the path, the states corresponds to different authorization first component is the transitivity flag, i.e. are you allowed to 
use simple imports, second component is the parent flag, i.e. can you look inside the parent scope. Transitions are either parent rule $P$, or imports flags 

\begin{figure}
\centering
\begin{tikzpicture}[shorten >=1pt,auto]

\tikzset{>=stealth',
    every edge/.append style={->}
  }
      \node[circle,draw] (TR) at (0,2.6) {\it 11};
      \node[circle,draw] (TX) at (4.5,5.2) {\it 10};
      \node[circle,draw] (NX) at (3,2.6) {\it 00};
      \node[circle,draw] (NR) at (4.5,0) {\it 01};
      \node (ri) at (4.5,2) {};
      
     \path 

     (TR) edge[loop left] node[left] {\small $\flag{x11} + P$} (TR)
     (TR.90) edge[bend left=35] node[above left] {\small $\flag{x10}$} (TX.175)
     (TR) edge[bend left=10] node[above] {\small $\flag{x00}$} (NX)
     (TR.280) edge[bend right=35] node[above right] {\small $\flag{x01}$} (NR.175)

     (TX) edge[loop above] node[above] {\small $\flag{x10}$} (TX)
          edge[bend left=10] node[right] {\small $\flag{x00}$} (NX)
     (TX.185) edge[bend right=35] node[below right] {\small $\flag{x11}$} (TR.80)
     (TX.340) edge[bend left=50] node[right] {\small $\flag{x01}$} (NR.20)

     (NX) edge[loop right] node[right] {\small $\flag{100}$} (NX)
          edge[bend left=10] node[left] {\small $\flag{110}$} (TX)
          edge[bend left=10] node[right] {\small $\flag{101}$} (NR)
     (NX) edge[bend left=10]  node[below] {\small $\flag{111}$} (TR)

     (NR) edge[loop below] node[below] {\small $\flag{101} + P$} (NR)
     (NR.30) edge[bend right=50] node[left] {\small $\flag{110}$} (TX.330)
     (NR) edge[bend left=10] node[left] {\small $\flag{100}$} (NX)
     (NR.185) edge[bend left=35] node[below left] {\small $\flag{111}$} (TR.270);

\end{tikzpicture}
%\rule{\textwidth}{.1pt}
\caption{Path Well Formedness Automata}
\label{fig:algoterm}
\end{figure}


\begin{figure}
\begin{boxedminipage}{\hsize}

\textbf{Edges in scope graph}

\smallskip

	\infrule{R}{
		x^p \in \R{S}
	}{
		\mathcal{I} \vdash x^p \scopea{x}{R(x^p)} S
	}
	
\smallskip

	\infrule{D}{
		d_x \in \D{S}
	}{
		\mathcal{I} \vdash S \scopea{x}{D(d_x)} d_x
  }

\smallskip

% 	\infrule{T}{}{
% 		\mathcal{I} \vdash S_1\cdot S_2\cdot S^* \scopea{x}{T} S_2 \cdot S^*
% 	}

% \smallskip

	\infrule{Par}{
		\P{S} = S'
	}{
		\mathcal{I} \vdash S \scopea{x}{Par} S'
	}

\smallskip

	\infrule{I}{
		y^p \in \mathcal{I}(S_1)  
		\tab 
		y^p \notin \mathcal{I} 
		\tab
		y^p\!,\mathcal{I} \vdash y^p \resolvea{P^*}\!\!\!d_y^{S_2}		
	}{
		\mathcal{I} \vdash S_1 \scopea{x}{I^t(y^p,P^*)} S_2 
	}

\smallskip

% 	\infrule{I}{
% 		y^p \in \I{S_1}  
% 		\tab 
% 		y^p \notin \mathcal{I} 
% 		\tab
% 		y^p\!,\mathcal{I} \vdash y^p \resolvea{P^*}\!\!\!d_y^{S_2^*}	
% 	}{
% 		\mathcal{I} \vdash S_1\cdot S_1^* \scopea{x}{I(y^p\!,P^*)} S_2^* 
% 	}

% \smallskip

\textbf{Path composition}

	\infrule{C1}{}{
		\mathcal{I} \vdash d_x \lra{[]}{x} d_x
	}

\smallskip

	\infrule{C2}{
		\mathcal{I} \vdash A \scopea{x}{P} B
		\tab 
		\mathcal{I} \vdash B \lra{P^*}{x} C
	}{
		\mathcal{I} \vdash A \lra{P\cdot P^*}{x} C
	}

\smallskip
\textbf{Resolution}

	\infrule{Res}{
 		\begin{array}{c}
    	\mathcal{I} \vdash x^p \lra{P^*_1}{x} d_x
			\tab
			\tab
			\tab
			P^*_1 : WFP
    	\\
  		\forall d'_x,P^*_2(\mathcal{I} \vdash x^p \lra{P^*_2}{x} d'_x\Rightarrow\neg
  	P^*_2 < P^*_1)
  	\end{array}
	}{
		\mathcal{I} \vdash x^p \resolvea{P^*_1} d_x
	}

\bigskip


\end{boxedminipage}
\caption{Resolution Calculus}
\label{fig:calculus}
\end{figure}


%\begin{figure}[p]

\begin{minipage}[t]{.49\hsize}
\begin{boxedminipage}[t]{\hsize}
\textbf{References and declarations}
\begin{itemize}
  \item $\mbox{$\dsi{x}{i}{S}$}$: declaration with name $x$ at position $i$ and optional
  associated named scope $S$
  \item $\ri{x}{i}$: reference with name $x$ 
  at position~$i$ 
\end{itemize}
\textbf{Scope graph}
\begin{itemize}
  \item $\G$: scope graph
  \item $\S{\G}$: scopes $S$ in $\G$ 
  \item $\D{S}$: declarations $\dsi{x}{i}{S'}$ in $S$
  \item $\R{S}$: references $\ri{x}{i}$ in $S$
  \item $\I{S}$: imports $\ri{x}{i}$ in $S$
  \item $\P{S}$: parent scope of $S$
\end{itemize}
\textbf{Well-formedness properties}
\begin{itemize}
  \item $\P{S}$ is a partial function
  \item The parent relation is well-founded
  \item Each $\ri{x}{i}$ and $\di{x}{i}$ appears in exactly one scope $S$
\end{itemize}
\end{boxedminipage}
\caption{Scope graphs}
\figurelabel{scope-graph}
\end{minipage}
\begin{minipage}[t]{.49\hsize}
\begin{boxedminipage}[t]{\hsize}
\textbf{Resolution paths}
\vspace*{-0.4\baselineskip}
$$\begin{array}{rl}
          s & := \dstep{\di{x}{i}}\ |\ \istep{\ri{x}{i}}{\dsi{x}{j}{S}}\ |\ \pstep\\
          p & := []\ |\ s\ |\ p\cdot p\\
          & \mbox{\rm (inductively generated)} \\[0pt]
          [] \cdot p & = p \cdot [] = p\\
          (p_1 \cdot p_2) &\cdot\ p_3  = p_1 \cdot (p_2 \cdot p_3)
\end{array}$$ 

\textbf{Well-formed paths}

\vspace*{-0.5\baselineskip}

\[
	   \WF(p) \Leftrightarrow p \in \pstep^* \cdot \istep{\_}{\_}^* 
\]
	
\textbf{Specificity ordering on paths}

%\bigskip

	\infrule{DI}{}{
		\dstep{\_} < \istep{\_}{\_}
	}

\medskip

	\infrule{IP}{}{
		\istep{\_}{\_} < \pstep 
	}

\medskip

	\infrule{DP}{}{
		 \dstep{\_} < \pstep
	}

\medskip

    \infrule{Lex1}{
		s_1 < s_2
	}{ 
		s_1\cdot p_1 < s_2 \cdot p_2
	}


\medskip

	\infrule{Lex2}{
		p_1 < p_2
	}{ 
		s \cdot p_1 < s \cdot p_2
	}

\smallskip

\end{boxedminipage}
\caption{Resolution paths, well-formedness predicate, and specificity
ordering.}
\figurelabel{order}
\end{minipage}

\bigskip

\begin{boxedminipage}{\hsize}
\textbf{Edges in scope graph}
\smallskip

	\infrule{P}{
		\P{S_1} = S_2
	}{
		\seeni \vdash \pstep : S_1 \edge S_2
	}

\medskip

	\infrule{I}{
		\ri{y}{i} \in \I{S_1}\setminus\seeni  
		\tab
		\seeni \vdash p : \ri{y}{i} \resolve \dsi{y}{j}{S_2}
	}{
		\seeni \vdash \istep{\ri{y}{i}}{\dsi{y}{j}{S_2}} : S_1 \edge S_2 
	}

\medskip
\textbf{Transitive closure}

%\medskip
       
	\infrule{N}{
		}{
		\seeni \vdash [] : A \medge A
	}

\medskip

	\infrule{T}{
		\seeni \vdash s : A \edge B
		\tab 
		\seeni \vdash p : B \medge C
	}{
		\seeni \vdash s \cdot p : A \medge C
	}

\smallskip

\textbf{Reachable declarations}
\medskip

	\infrule{R}{
		\di{x}{i} \in \D{S'}
		\tab
		\seeni \vdash p : S \medge S'
		\tab 
		\WF(p)
	}{
		\seeni \vdash p \cdot \dstep{\di{x}{i}} : S \reach \di{x}{i}
	}

\medskip

\textbf{Visible declarations}
\medskip

\infrule{V}{
 	\seeni \vdash p : S \reach \di{x}{i}
			\tab
			\tab
			% tab
  		\forall j,p' (
  		   \seeni \vdash p' : S \reach \di{x}{j} \Rightarrow 
  		   \neg (p' < p)
  		)
  	}{
		\seeni \vdash {p} : S \resolve \di{x}{i}
	}

\medskip

\textbf{Reference resolution}

\medskip 

\infrule{X}{
		\ri{x}{i} \in \R{S}
    \tab
    \{ \ri{x}{i} \} \cup \seeni \vdash p : S \resolve \di{x}{j}
	}{
		\seeni \vdash p : \ri{x}{i} \resolve \di{x}{j}
	}
\end{boxedminipage}
\caption{Resolution calculus}
\figurelabel{rescalc}



\end{figure}




Set of \emph{seen-imports} : 
\begin{itemize}
 \item how do we introduce it ?
 \item in which cases is it really necessary ?
\end{itemize}

%%% Local Variables: 
%%% mode: latex
%%% TeX-master: "../document"
%%% End: 
