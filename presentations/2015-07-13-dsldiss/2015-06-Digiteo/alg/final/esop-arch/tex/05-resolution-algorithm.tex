\section{Resolution Algorithm}
\sectionlabel{resalg}

The calculus of \Section{rescalc} gives a precise definition of resolution.
In principle, we can search for derivations in the calculus 
to answer questions such as ``Does this variable
reference resolve to this declaration?'' or ``Which variable declarations
does this reference resolve to?''  But automating this search process is not trivial, 
because of the need for back-tracking and 
because the paths in reachability derivations can have cycles (visiting
the same scope more than once), and hence can grow arbitrarily long.

In this section we describe a deterministic and terminating
\emph{algorithm} for computing resolutions,
which provides a practical basis for implementing tools based on scope graphs,
and prove that it is sound and complete with respect to the calculus.
This algorithm also connects the calculus, which talks about
resolution of a single variable at a time, to more conventional descriptions of
binding which use ``environments'' or ``contexts'' to describe \emph{all}
the visible or reachable declarations accessible from a program location.

For us, an \emph{environment} is just a set of declarations $\di{x}{i}$. 
This can be thought of as a function from identifiers to (possible empty) 
sets of declaration positions.  (In this paper, we leave the representation 
of environments abstract; in practice, one would use a hash table or other
dictionary data structure.) We construct an atomic environment corresponding
to the declarations in each scope, 
and then combine atomic environments to describe the sets of 
reachable and visible declarations resulting from the parent and import 
relations. The key operator for combining environments is {\it shadowing},
which returns the union of the declarations in two environments restricted so
that if a variable $x$ has any declarations in the first environment, no
declarations of $x$ are included from the second environment.  More formally:


\begin{figure}[t]
\renewcommand{\S}{\mathcal{S}}
\begin{boxedminipage}{\hsize}
\vspace*{-0.5\baselineskip}
$$
\begin{array}{ll}
  \Res{\seeni}{\ri{x}{i}} & := \{ \di{x}{j} \mid \exists S\ s.t.\ \ri{x}{i} \in \R{S} \wedge \di{x}{j} \in \Env{V}{\{\ri{x}{i}\} \cup \seeni}{\emptyset}{S} \} \smallskip  \\
  \Env{V}{\seeni}{\seens}{S}  & := \Env{L}{\seeni}{\seens}{S} \triangleleft \Env{P}{\seeni}{\seens}{S} \smallskip\\
 \Env{L}{\seeni}{\seens}{S} & := \Env{D}{\seeni}{\seens}{S} \triangleleft \Env{I}{\seeni}{\seens}{S} \smallskip\\
 \Env{D}{\seeni}{\seens}{S} &:= \left\{
    \begin{array}{l}
      \emptyset  \text{ if } S\in\seens\\
      \D{S}\smallskip\\
    \end{array}
 \right.\smallskip\\
 \Env{I}{\seeni}{\seens}{S} &:= \left\{
    \begin{array}{l}
      \emptyset  \text{ if } S\in\seens\\
      \bigcup \left\{ \Env{L}{\seeni}{\{S\}\cup\seens}{S_y} \mid \ri{y}{i} \in \I{S} \setminus 
\seeni \wedge \dsi{y}{j}{S_y} \in \Res{\seeni}{\ri{y}{i}}\right\}\smallskip\\
%       \bigcup \limits_{\r{y} \in \I{S} \setminus \seeni} 
%          \left\{ \Env{L}{\seeni}{\{S\}\cup\seens}{S_y} \mid \ds{y}{S_y} \in \Res{\seeni}{\r{y}}\right\}\smallskip\\
    \end{array}
 \right.\smallskip\\
 \Env{P}{\seeni}{\seens}{S} & := \left\{
    \begin{array}{l}
      \emptyset  \text{ if } S\in\seens\\
      \Env{V}{\seeni}{\{S\}\cup\seens}{\P{S}}
    \end{array}
 \right.\smallskip\\
\end{array}\medskip
$$
\vspace*{-1.5\baselineskip}
\end{boxedminipage}
\vspace*{-\baselineskip}
\caption{Resolution algorithm}
\label{fig:resalg}
\end{figure}


% \vspace*{-0.3\baselineskip}

\begin{definition}[Shadowing] For any environments $E_1$, $E_2$, we write:\\
\centerline{$E_1 \hiding E_2 := E_1 \cup \{ \di{x}{i} \in E_2 \mid  \nexists\ \di{x}{i'} \in
E_1\}$}
\end{definition}

%\vspace*{-.2\baselineskip}

\noindent
Figure~\ref{fig:resalg} specifies an algorithm $\Res{\seeni}{\ri{x}{i}}$
for resolving a reference $\ri{x}{i}$
to a set of corresponding declarations $\di{x}{j}$.
Like the calculus,
the algorithm avoids trying to use an import to resolve itself
by maintaining a set $\seeni$ of ``already seen'' imports.
The algorithm works by computing the full environment $\Env{V}{\seeni}{\seens}{S}$ 
of declarations that are visible in the scope $S$ containing $\ri{x}{i}$, and 
then extracting just the declarations for $x$. The full environment, in turn,
is built from the more basic environments $\Envu{D}$ of immediate declarations,
$\Envu{I}$ of imported declarations, and $\Envu{P}$ of lexically enclosing declarations,
using the shadowing operator.  The order of construction matches both the $\WF$ restriction
from the calculus, which prevents the use of parent after an import, and  the path ordering $<$,
which prefers immediate declarations over imports and imports over 
declarations from the parent scope.  
(Note that the algorithm does \emph{not} work for the variants of $\WF$
and $<$ described in \Section{extensions}.)
%The algorithm is naturally implemented by 
%constructing the various environments lazily. 
A key difference from the calculus is that the shadowing operator is applied at
each stage in environment construction, rather than applying the visibility 
criterion just once at the ``top level'' as in calculus rule $V$. 
This difference is a natural consequence of the fact that the algorithm computes
sets of declarations rather than full derivation paths, so it does not maintain
enough information to delay the visibility computation.

\paragraph{Termination} The algorithm is terminating using the well-founded
lexicographic measure  $(|\R{\G} \setminus \seeni|,|\S{\G} \setminus \seens|)$.
Termination is straightforward by unfolding the calls to $Res$ in $\Envu{I}$ and then 
inlining the definitions of $\Envu{V}$ and $\Envu{L}$: this gives an equivalent algorithm 
% (presented in Appendix \refproofappendix~of \cite{TUD-SERG-2015-001-local}) 
in which the measure strictly decreases at every recursive call.

\subsection{Correctness of Resolution Algorithm}

The resolution algorithm is sound and complete
with respect to the calculus. 
\begin{theorem}
\label{theorem:correctness}
$\forall\ \seeni, \ri{x}{i}, j, (\di{x}{j}\in \Res{\seeni}{\ri{x}{i}}) 
\iff (\exists p\ s.t.\ \seeni \vdash p : \ri{x}{i} \resolve \di{x}{j})$.
\end{theorem}

We sketch the proof of this theorem here; 
details of the supporting lemmas and proofs are in Appendix  \refproofappendix~of
\cite{TUD-SERG-2015-001-local}.
To begin with, we must deal with the fact that the calculus can
generate reachability derivations with cycles,
but the algorithm does not follow cycles.
In fact, \emph{visibility} derivations cannot have cycles: 

\begin{lemma}%[Resolution paths are cycle-free]
\label{lemma:cycle-free}
If $\seeni \vdash p : \ri{x}{i} \resolve \di{x}{j}$ then $p$ is cycle-free.
\end{lemma}

\begin{figure}[t]
\begin{boxedminipage}{\hsize}
\textbf{Transitive closure}
       
\vspace*{-0.5\baselineskip}

	\infrule{N'}{
		}{
		\seeni,\seens \vdash [] : A \medge A
	}

\medskip

	\infrule{T'}{
		\seeni \vdash s : A \edge B
		\tab 
                B \not\in \seens
                \tab
		\seeni,\{B\} \cup \seens \vdash p : B \medge C
	}{
		\seeni,\seens \vdash s \cdot p : A \medge C
	}

\smallskip

\textbf{Reachable declarations}
\medskip

	\infrule{R'}{
                \di{x}{i} \in \D{S'}
                \tab
                S \not\in \seens
                \tab
                \seeni,\{S\} \cup \seens \vdash p : S \medge S'
		\tab 
		\WF(p)
	}{
		\seeni,\seens \vdash p \cdot \dstep{\di{x}{i}} : S \reach \di{x}{i}
	}

\medskip

\textbf{Visible declarations}
\medskip

\infrule{V'}{
 	\seeni,\seens \vdash p : S \reach \di{x}{i}
			\tab
			\tab
			% tab
  		\forall j,p' (
  		   \seeni,\seens \vdash p' : S \reach \di{x}{j} \Rightarrow 
  		   \neg (p' < p)
  		)
  	}{
		\seeni,\seens \vdash p : S \resolve \di{x}{i}
	}

\medskip

\textbf{Reference resolution}

\medskip 

\infrule{X'}{
		\ri{x}{i} \in \R{S}
                \tab
                \{\ri{x}{i}\} \cup \seeni,\emptyset \vdash p : S \resolve \di{x}{j}
	}{
		\seeni \vdash p : \ri{x}{i} \resolve \di{x}{j}
	}

\medskip

\end{boxedminipage}
\vspace*{-\baselineskip}
\caption{``Primed'' resolution calculus with ``seen scopes'' component}\medskip
\label{fig:seencalc}
\end{figure}



\noindent
We therefore begin 
% our attack on the proof of Theorem~\ref{theorem:correctness}
by defining an alternative version of the calculus that
prevents construction of cyclic paths.  This alternative calculus 
consists of the original rules $(P),(I)$ from Figure~\ref{fig:rescalc} 
together with the new rules $(N'),(T'),(R'),(V'),(X')$ 
from Figure~\ref{fig:seencalc}.
The new rules describe transitions that include a ``seen scopes'' component $\seens$ which
is used to enforce acyclicity of paths. By inspection, this is the only difference 
between the ``primed'' system and original one.
Thus, by Lemma~\ref{lemma:cycle-free}, we have

\begin{lemma}
\label{lemma:primed}
$\forall\seeni, \seens, \di{x}{i}, (\exists p\ s.t.\ \seeni \vdash p:S \resolve
\di{x}{i})\!\iff\!(\exists p\ s.t.\ \seeni,\emptyset \vdash p:S \resolve \di{x}{i})$.
\end{lemma}

\noindent
Hereinafter, we can work with the primed system.

Next we define a family of sets $\pathx{}$ of derivable paths in the (primed) calculus. 
\begin{definition}[Path Sets]\vspace*{-.3\baselineskip}
  \begin{center}
$
\begin{array}{lll}
  \paths{D}{\seeni}{\seens}{S} & := & \{ p \mid \exists\ \di{x}{i}\ s.t.\  p = \dstep{\di{x}{i}} \wedge \seeni,\seens \vdash p : S \reach \di{x}{i}\}\\
  \paths{P}{\seeni}{\seens}{S} & := & \{ p \mid \exists\ p'\ \di{x}{i}\ s.t.\ p = \pstep \cdot p'\wedge \\
  & & \tab\tab\tab \seeni,\seens \vdash p : S \reach \di{x}{i} \wedge \seeni,\{S\}\cup\seens \vdash p' : \P{S} \resolve \di{x}{i} \} \\
  \paths{I}{\seeni}{\seens}{S} & := & \{ p \mid \exists\ p'\ \di{x}{i}\ \ri{y}{j}\ \dsi{y}{j'}{S'}\ s.t.\ p = \istep{\ri{y}{j}}{\dsi{y}{j'}{S'}}  \cdot p' \wedge \\
  & & \tab\tab\tab\seeni,\seens \vdash p : S \reach \di{x}{i} \wedge \seeni,\{S\}\cup\seens \vdash p' : S' \resolve \di{x}{i} \}
\\
  \paths{L}{\seeni}{\seens}{S} & := &  \{ p \mid  \exists\ \di{x}{i}\ s.t.\ \seeni,\seens \vdash p : S \resolve \di{x}{i} \wedge p \in \istep{\_}{\_}^*\cdot \dstep{\_}\}\\
  \paths{V}{\seeni}{\seens}{S} & := &  \{ p \mid  \exists\ \di{x}{i}\ s.t.\ \seeni,\seens \vdash p : S \resolve \di{x}{i} \}\\
\end{array}
$    
  \end{center}
\label{def:pathsets}
\end{definition}
\vspace*{-.4\baselineskip}
These sets are designed to correspond to the various classes of environments $\Envu{C}$.
$\pathx{D}$, $\pathx{P}$, and $\pathx{I}$ contain all reachability 
derivations starting with a $\dstep{\_}$, $\pstep$, or $\istep{\_}{\_}$ respectively, 
with the further condition that the \emph{tail} of each derivation is 
a visibility derivation (i.e. is most specific among all reachability derivations).
$\pathx{V}$ describes the set of all visibility derivations. ($\pathx{L}$ is similar, but omits paths including $\pstep$ steps, because well-formedness 
prevents using these steps after an import step.) For compactness, we state the key result uniformly over all classes of sets:
\begin{definition} For any path $p$, $\defof{p} := \di{x}{i}$ iff $\exists p'\ s.t.\ p = p' \cdot \dstep{\di{x}{i}}$
and for any set of paths $P$, $\defsof{P} := \{ \defof{p} \mid p \in P\}$.
\end{definition}

\begin{lemma}
\label{lemma:pathset-alg}
For each class $C \in \{V,L,D,I,P\}$:\vspace*{-.3\baselineskip}
$$ \forall\ \seeni\ \seens\ S, \Env{c}{\seeni}{\seens}{S} = \defsof{\paths{C}{\seeni}{\seens}{S}}$$
\end{lemma}

% One main difference between the calculus and the resolution algorithm is in the minimality check. The calculus first computes all the reachable definitions 
% from a current scope before choosing the minimal ones as the visible ones whereas the calculus minimize at every step during the path computation. 
% In order to bridge this gap we first prove that the tail of a visibility or reachability path is also a
% valid path starting at the next scope.

\begin{proof} 
We first prove two auxiliary lemmas about reachability and visibility after one step:
\begin{multline}\label{eqn:tailreach}\tag{$\lozenge$}
   \forall\ \seeni\ \seens\ s\ p\ S\ \di{x}{i},
   (\seeni,\seens \vdash s \cdot p\cdot\dstep{\di{x}{i}} : S \reach \di{x}{i} \Longrightarrow
   \seeni,\{S\}\cup\seens \vdash s : S \edge S' \Longrightarrow\\ 
   \seeni,\{S\}\cup\seens \vdash p\cdot\dstep{\di{x}{i}} : S' \reach \di{x}{i})    
  \end{multline}\vspace*{-8mm}
\begin{multline}\label{eqn:tailvis}\tag{$\blacklozenge$}
  \forall\ \seeni\ \seens\ s\ p\ S\ \di{x}{i},
  (\seeni,\seens \vdash s \cdot p : S \resolve \di{x}{i} \Longrightarrow 
  \seeni,\{S\}\cup\seens \vdash s : S \edge S' \Longrightarrow\\ 
  \seeni,\{S\}\cup\seens \vdash p : S' \resolve \di{x}{i})  
\end{multline}
Then we proceed by three nested inductions, 
the outer one on $\seeni$ (or, more strictly, on \mbox{$|\R{\G} \setminus \seeni|$}, the number of references
\emph{not} in $\seeni$), the second one on $\seens$ (more strictly, on \mbox{$|\S{\G} \setminus \seens|$}, the number of 
scopes \emph{not} in $\seens$) and the third one on the class $C$ with the order $V > L > {P,I,D}$. Then we conclude using \ref{eqn:tailreach} and \ref{eqn:tailvis} and a number of other technical results. Details are in Appendix~\refproofappendix\ of \cite{TUD-SERG-2015-001-local}. \qed
\end{proof}

With these lemmas in hand we proceed to prove Theorem~\ref{theorem:correctness}.
\begin{proof} Fix $\seeni$, $\ri{x}{i}$, and $j$. Given $S$, the (unique) scope such that $\ri{x}{i} \in \R{S}$:\smallskip\\
\centerline{$\di{x}{j} \in \Res{\ri{x}{i}}{\seeni} \Leftrightarrow \di{x}{j} \in \Env{V}{\{\ri{x}{i}\} \cup \seeni}{\emptyset}{S}$\smallskip}
By the $V$ case of Lemma~\ref{lemma:pathset-alg} and the definition of $\pathx{S}$, this is equivalent to\smallskip\\
\centerline{$\exists p\ s.t.\ \{\ri{x}{i}\} \cup \seeni,\emptyset \vdash p : S \resolve \di{x}{j}$\smallskip}
which, by Lemma~\ref{lemma:primed} and rule $X$, is equivalent to $\exists p\ s.t.\ \seeni \vdash p : \ri{x}{i} \resolve \di{x}{j}$. \qed 
\end{proof}

\endinput

We have two key lemmas connecting the sets $\pathx{}$ with the calculus on the one hand
and the algorithm on the other.  First, $\pathx{V}$ consists of exactly the resolution 
paths described by the calculus:

\begin{lemma}[Resolution and $\pathx{V}$ equivalence] 
\label{lemma:distr}
$$\forall\ \seeni\ \seens\ S\ p\ \d{x},
 p \cdot \dstep{\d{x}}\in\paths{V}{\seeni}{\seens}{S} \iff \seeni,\seens \vdash p \cdot \dstep{\d{x}}: S \resolve \d{x}$$
\end{lemma}
\begin{proof}\vspace*{-.3\baselineskip} 
We first prove two auxiliary lemmas about reachability and visibility after one scope transition:
\vspace*{-3mm}
 \begin{multline}\label{eqn:tailreach}\tag{$\lozenge$}
   \forall\ \seeni\ \seens\ s\ p\ S\ d\ s.t.\ 
   \seeni,\seens \vdash s \cdot p\cdot\dstep{\d{x}} : S \reach \d{x} \Longrightarrow\\
   \seeni,\{S\}\cup\seens \vdash s : S \edge S' \Longrightarrow 
   \seeni,\{S\}\cup\seens \vdash p\cdot\dstep{\d{x}} : S' \reach \d{x}    
  \end{multline}\vspace*{-9mm}
\begin{multline}\label{eqn:tailvis}\tag{$\blacklozenge$}
\forall\ \seeni\ \seens\ s\ p\ S\ d\ s.t.\ 
\seeni,\seens \vdash s \cdot p : S \resolve \d{x} \Longrightarrow \\ 
\seeni,\{S\}\cup\seens \vdash s : S \edge S' \Longrightarrow 
\seeni,\{S\}\cup\seens \vdash p : S' \resolve \d{x}  
\end{multline}
Then we proceed by case analysis on the first step of $p = s\cdot p'$. 

($\Rightarrow$) The first step $s$ tells which set among $\pathx{D}$,$\pathx{P}$,$\pathx{I}$ $p$ comes from.
Thus we have $\seeni,\seens \vdash p \cdot \dstep{\d{x}}: S \reach \d{x}$ and must prove that $p$ is minimal.
Assume $\bar{s}\cdot \bar{p} < s\cdot p'$. Then using definition of \visible\ and  \ref{eqn:tailreach} we can prove that $\bar{p}$ is a reachable path smaller than $p'$, contradicting the minimality of $p'$.\medskip

($\Leftarrow$) The lemma \ref{eqn:tailvis} provides the resolution of the tail which allows us to place $p$ in
the corresponding set $\pathx{D}$,$\pathx{P}$,$\pathx{I}$. Then knowing that $p$
is minimal we lift it through the \visible\ calls to $\pathx{V}$. Details are in Appendix~\refproofappendix of \cite{TUD-SERG-2015-001-local}.

\end{proof}

\noindent
Secondly, $\pathx{}$ sets correspond to $\Envu{}$ sets. For compactness, we state this result
uniformly over all classes of sets:
\begin{lemma}[Algorithm and path sets]
\label{lemma:pathset-alg}
For each class $C \in \{V,L,D,I,P\}$:\vspace*{-.3\baselineskip}
$$ \forall\ \seeni\ \seens\ S, \Env{c}{\seeni}{\seens}{S} = \defsof{\paths{C}{\seeni}{\seens}{S}}$$
\end{lemma}
\begin{proof}{By two nested inductions, 
the outer one on $\seeni$ (or, more strictly, on \mbox{$|\R{\G} \setminus \seeni|$}, the number of references
\emph{not} in $\seeni$) and the inner one on $\seens$ (more strictly, on \mbox{$|\S{\G} \setminus \seens|$}, the number of 
scopes \emph{not} in $\seens$).  We then proceed by cases on the class $C$, using
Lemmas~\ref{lemma:min} and \ref{lemma:distr} and a number of other technical results. 
Details are in Appendix~\refproofappendix of \cite{TUD-SERG-2015-001-local}.}
\end{proof}

\noindent
% Proof of Theorem~\ref{theorem:correctness} is now straightforward by composing Lemmas \ref{lemma:pathset-alg}, \ref{lemma:distr} and \ref{lemma:primed}.
With these lemmas in hand we proceed to prove Theorem~\ref{theorem:correctness} .
\begin{proof} Fix $\seeni$,$\r{x}$, and $\d{x}$. Given $S$, the (unique) scope such that $\r{x} \in \R{S}$:\smallskip\\
\centerline{$\d{x} \in \Res{\r{x}}{\seeni} \Leftrightarrow \d{x} \in \Env{V}{\{\r{x}\} \cup \seeni}{\emptyset}{S}$\smallskip}
By the $V$ case of Lemma~\ref{lemma:pathset-alg} and definition of  this is equivalent to\smallskip\\
\centerline{$\exists p\ s.t.\ p \cdot \dstep{\d{x}} \in \paths{S}{\{\r{x}\} \cup \seeni}{\emptyset}{S}$\smallskip}
By Lemma~\ref{lemma:distr}, this is equivalent to:\smallskip\\
\centerline{$\exists p\ s.t.\ \{\r{x}\} \cup \seeni,\emptyset \vdash p : S \resolve \d{x}$\smallskip}
which, by Lemma~\ref{lemma:primed} and rule $X$, is equivalent to $\exists p\ s.t.\ \seeni \vdash p : \r{x} \resolve \d{x}$. \qed 
\end{proof}


%%% Local Variables: 
%%% mode: latex
%%% TeX-master: "../document"
%%% End: 

