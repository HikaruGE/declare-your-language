\section{Proofs}
\label{app:proofs}

We often write just $\r{x}$ (resp. $\d{x}$) for a reference (resp. declaration) at some particular but unnamed position.

%\renewcommand{\scopea}[1]{\stackrel{#1}{\longrightarrow}}
\renewcommand{\labelenumi}{\tab $\roman{enumi})$ }

\subsection{Termination of resolution algorithm}\label{subsection:inlinedresalg}

Figure \ref{fig:inlinedresalg} presents the inlined resolution algorithm that only use strictly 
decreasing recursive calls.

\begin{figure}[h]
\renewcommand{\S}{\mathcal{S}}
\begin{boxedminipage}{\hsize}
$
\begin{array}{ll}
  \Res{\seeni}{\r{x}} & := \left\{ \di{x}{i} \left| \exists S\ s.t.\ \r{x} \in \R{S} \wedge \di{x}{i} \in 
  \left(\begin{array}{l}
    \Env{D}{\{\r{x}\} \cup \seeni}{\emptyset}{S}  \\
     \triangleleft ~\Env{I}{\{\r{x}\} \cup \seeni}{\emptyset}{S}  \\
     \triangleleft ~\Env{P}{\{\r{x}\} \cup \seeni}{\emptyset}{S}    
  \end{array}\right)
 \right\}\right. \smallskip  \\
\Env{D}{\seeni}{\seens}{S} &:= \left\{
    \begin{array}{l}
      \emptyset  \text{ if } S\in\seens\\
      \D{S}\smallskip\\
    \end{array}
 \right.\smallskip\\
 \Env{I}{\seeni}{\seens}{S} &:= \left\{
    \begin{array}{l}
      \emptyset  \text{ if } S\in\seens\\
      \bigcup \left\{
        \begin{array}{l}
          \Env{D}{\seeni}{\{S\}\cup\seens}{S_y}\\
          \triangleleft ~\Env{I}{\seeni}{\{S\}\cup\seens}{S_y}
        \end{array}\hspace*{-1mm}
      \left| 
        \begin{array}{l}
          \r{y} \in \I{S} \setminus \seeni\ \wedge \\
          \exists S_y\ s.t.\ \r{y} \in \R{S_y}\ \wedge \\
          \ds{y}{S_y} \in 
          \left(\begin{array}{l}
              \Env{D}{\{\r{x}\} \cup \seeni}{\emptyset}{S_y}  \\
              \triangleleft ~\Env{I}{\{\r{x}\} \cup \seeni}{\emptyset}{S_y}  \\
              \triangleleft ~\Env{P}{\{\r{x}\} \cup \seeni}{\emptyset}{S_y}    
            \end{array}\right)
        \end{array}\hspace*{-1.5mm}
      \right\}\right.\smallskip\\
    \end{array}
 \right.\smallskip\\
 \Env{P}{\seeni}{\seens}{S} & := \left\{
   \begin{array}{l}
     \emptyset  \text{ if } S\in\seens\\
     \left(\begin{array}{l}
       \Env{D}{\seeni}{\{S\}\cup\seens}{\P{S}} \\
       \triangleleft ~\Env{I}{\seeni}{\{S\}\cup\seens}{\P{S}} \\
       \triangleleft ~\Env{P}{\seeni}{\{S\}\cup\seens}{\P{S}} 
     \end{array}\right)
   \end{array}
 \right. \smallskip\\
\end{array}\medskip
$
\end{boxedminipage}
\caption{Inlined resolution algorithm}
\label{fig:inlinedresalg}
\end{figure}

\subsection{Well-formed Paths and Specificity Ordering}

We can give a simple grammatical characterization of well-formed paths, and
use it to define a measure on paths that is compatible with the specificity 
ordering.

\begin{lemma}[Well-formed path measure]\label{lem:wfpm} For all well-formed paths $p$, there
exist $i,j \geq 0$ such that $p = \pstep^i \cdot \istep{\_}{\_}^j$. 
We write
$m(p)$ for the corresponding pair $(i,j)$ and write $<$ for the lexicographic ordering
on these pairs.
\end{lemma}

\begin{lemma}[WF path measure and specificity ordering are compatible]\label{lem:pathmescomp} 
For all well-formed paths $p,p'$,
\begin{equation*}
 p\cdot\dstep{\_} < p'\cdot\dstep{\_} \iff m(p) < m(p')  
\end{equation*}
\end{lemma}
\begin{proof} Let $m(p) = (i,j)$ and $m(p') = (i',j')$. Let
$q = p\cdot\dstep{\_}$ and $q' = p'\cdot\dstep{\_}$,
and write $q_k,q'_{k}$ for the $k$-th steps of $q$ and $q'$, respectively.
\\ ($\Longrightarrow$) Assume $p < p'$.
Then in the first step $k$ where $q$ and $q'$ differ (i.e., the first application of rule $Lex1$),
we have three possible cases:
\begin{itemize}
 \item ($DI$) $q_k = \dstep{\_}$ and $q'_k = \istep{\_}{\_}$: then $i = i'$ and $j < j'$; hence $m(p) < m(p')$.
 \item ($IP$) $q_k = \istep{\_}{\_}$ and $q'_k = \pstep$: then $i < k \le i'$; hence $m(p) < m(p')$.
 \item ($DP$) $q_k = \dstep{\_}$ and $q'_k = \pstep$: then $i < k \le i'$; hence $m(p) < m(p')$. 
\end{itemize}
($\Longleftarrow$) Suppose $m(p) < m(p')$. Then there are two cases:
\begin{itemize}
 \item $i < i'$: then $\forall k \leq i, q_k = q'_k$,  
                    $q_{i+1} = \istep{\_}{\_}$ or $\dstep{\_}$, and $q'_{i+1} = \pstep$; hence $p < p'$.
 \item $i = i'$ and $j < j'$: then  $\forall k \leq i+j, q_k = q'_k$,
                    $q_{i+j+1} = \dstep{\_}$ and $q'_{i+j+1} = \istep{\_}{\_}$; hence $p < p'$.
\end{itemize}
 \qed
\end{proof}


\subsection{Cycles in Resolution Paths}

To enforce the termination of the resolution algorithm, we only look for cycle-free paths in this algorithm.
Looking for cycle-free paths is sufficient:

\begin{definition}[Cycle] 
\label{def:cycle}
A path $p$ contains a cycle if there exist $p_1, p_2, p_3$ 
such that $p = p_1 \cdot p_2 \cdot p_3$, $p_2 \neq []$, and 
there exist $A$, $B$ and $S$ such that 
$\seeni \vdash p_1 : A \medge S$, $\seeni \vdash p_2 : S \medge S$,and
$\seeni \vdash p_3: S \medge B$.
\end{definition}

\begin{lemma}[Lemma~\ref{lemma:cycle-free}: Resolution paths are cycle-free.]
If 
\begin{equation*}
\seeni \vdash p : \r{x} \resolve \d{x}
\end{equation*}
then $p$ is cycle-free.
\end{lemma}
\begin{proof}
By rule $(X)$, we have $\seeni' \vdash p: S_r \resolve \d{x}$ where
$\seeni'$ = $\{\r{x}\} \cup \seeni$ and
$S_r$ is the (unique) scope such that $\r{x} \in \R{S_r}$.
By rule $(V)$ we have $\seeni' \vdash p: S_r \reach \d{x}$ and
\begin{equation}\label{eqn:noshorter}\tag{\mbox{$*$}}
\nexists\ p'\ \di{x}{i'}\ s.t.\ \seeni' \vdash p': S_r \reach \di{x}{i'}\ \mbox{\it and}\ p' < p.
\end{equation}
By rule $(R)$ we have 
$\seeni' \vdash p_0: S_r \medge S_d$ where $p = p_0 \cdot \dstep{\d{x}}$ and
$S_d$ is the (unique) scope such that $\d{x} \in \D{S_d}$;
moreover, $\WF(p_0)$ holds, so by Lemma~\ref{lem:wfpm}, 
$p_0=\pstep^i\cdot\istep{\_}{\_}^j$ for some $i,j \geq 0$.
We claim $p_0$ contains no cycle.
For suppose it does, with $p_1, p_2, p_3$ and $S$ as in Definition~\ref{def:cycle}, 
and consider the {\it shrunken} path $p' = p_1 \cdot p_3$. Then, since
$\seeni' \vdash p_1 : S_r \medge S$ and $\seeni' \vdash p_3 : S \medge S_d$, we have
$\seeni' \vdash p' : S_r \medge S_d$.  Also, $p'$ must have the form $\pstep^{i'}\cdot\istep{\_}{\_}^{j'}$,
where, depending on the position of $p_2$ in $p_0$,  
either (i) $i' < i$ or (ii) $i' = i$ and $j' < j$. Therefore $\WF(p')$ holds, 
and we have $\seeni' \vdash p' \cdot \dstep{\d{x}} : S_r \reach \d{x}$.
But since $m(p') = (i',j') < (i,j) = m(p_0)$, then by Lemma~\ref{lem:pathmescomp}, $p'\cdot\dstep{\d{x}} < p_0\cdot\dstep{\d{x}}$, contradicting ($\ref{eqn:noshorter}$). 
So $p_0$ has no cycle, and hence neither does $p$. \qed
\end{proof}

\begin{comment}

\begin{lemma}[Cycle-free Path Existence]
\label{lemma:cycle-free} 
If there is a path for a resolution then there is
a cycle-free path for this resolution.
\vspace*{-.6\baselineskip}
$$
 \forall\seeni,\r{x},\d{x},  (\exists p\ s.t.\ \seeni \vdash p : \r{x} \resolve \d{x}) \Longrightarrow 
(\exists p, p\ \mbox{\it is cycle-free} \wedge \seeni \vdash p : \r{x} \resolve
\d{x}) $$
\end{lemma}
\begin{proof}
The set of all resolution paths 
$\{ p \mid \seeni \vdash p\cdot\dstep{\d{x}} : \r{x} \resolve \d{x}\}$
must contain (at least one) shortest path $p$ (we use \emph{shorter} when referring to  the length of the path and \emph{smaller} when referring to the path ordering $<$).  
We claim $p$ contains no cycle.
For suppose it does, with $p_1, p_2, p_3$ as in Definition~\ref{def:cycle} and 
consider the {\it shrunken} path $p' = p_1 \cdot p_3$. Then:
\begin{equation*}
\r{x} \in S_r \wedge \d{x} \in S_d \wedge \seeni \vdash p' : S_r \medge S_d
\end{equation*}
Moreover, by considering all possible patterns of steps for $p_2$, 
we can see that $p'$ is well-formed. Hence
\begin{equation*}
\seeni \vdash p' : S_r \reach \d{x}  
\end{equation*}
We claim $p'$ is also a minimal (most-specific) path resolving $\r{x}$ under $\seeni$.
For suppose not. Then there is a $q$ such that:
\begin{equation*}
  \seeni \vdash q : \r{x} \reach \d{x} \wedge q < p_1 \cdot p_3 
\end{equation*}
Then we have two cases:
  \begin{itemize}
   \item $q= q_1 \cdot q_2 $ and $q_1 < p_1$; and therefore $q < p_1\cdot p_2 \cdot p_3 = p$   
   \item $q = p_1 \cdot q_2 $ and $q_2 < p_3$; in this case $p_1 \cdot p_2
   \cdot q_2$ is also a well-formed path more specific than $p_1\cdot p_2 \cdot p_3 = p$,
  \end{itemize}
In either case, we have exhibited a well-formed path more specific than $p$, contradicting
the assumed minimality of $p$. 
Therefore  $p'$ is indeed minimal, so we have:
\begin{equation*}
\seeni \vdash p' : \r{x} \resolve \d{x} 
\end{equation*}
But $p'$ is strictly shorter than $p$, which is a contradiction. 
So $p$ must not contain a cycle. \qed
\end{proof}
\end{comment}

\subsection{Shadowing and Resolution}

%We present here a lemma that allows us to exhibit a minimal (i.e., most specific) path for the resolution of a reference
%as soon as we know that some declaration is reachable from the reference. 
%In proofs of minimality of $p$, it will allow us not only to suppose (for contradiction) that
%$p$ is not minimal (therefore there is more specific path $p'$ such that $\seeni,\seens \vdash p' : S \reach \d{x}$) 
%but that there is a more specific path $p'$ such that $\seeni,\seens \vdash p' : S \resolve \d{x}$.
%Using this measure we can directly prove the existence of a minimal well-formed resolution path 
%when a well-formed resolution path exists. This means that as soon as a declaration is reachable from a reference, it can only be shadowed
%by other declarations and that this shadowing process stops somewhere. 
%APT: I couldn't find a way to clarify this paragraph to my satisfaction...


\begin{lemma}[Shadowing preserves resolution] \label{lemma:shadowexists}
If there is a reachability path from a scope to the declaration of a variable, then there
is a visibility path, at least as specific, from that scope to some declaration of 
that variable.
\begin{multline*}
\forall\ \seeni\ \seens\ S\ p\ \di{x}{i},
\seeni,\seens \vdash p : S \reach \di{x}{i} \Longrightarrow\\
(\exists\ q\ j\ s.t.\ \seeni,\seens \vdash q : S \resolve \di{x}{j} \wedge q \leq p).
\end{multline*}
\end{lemma}
\begin{proof} Fix $\seeni$, $\seens$, $S$, $p$, and $\di{x}{i}$. Let 
$W := \{ p' \mid \exists i'\ s.t.\ \seeni,\seens \vdash p' : S \reach \di{x}{i'} \wedge p' \le p\}$
be the set of reachability paths from $S$ to $x$ that are at least as specific as $p$. 
We need to show that $W$ contains a minimal reachable path, i.e., an element $q$ such that:
$\forall i'\ p', \seeni,\seens \vdash p' : S \reach \di{x}{i'} \Rightarrow \neg (p' < q)$.
$W$ is not empty (it contains $p$) and the set of associated measures of 
its members, i.e.
\begin{equation*}
 m(W) = \{ m(p') \mid \exists\ i'\ s.t.\ \seeni \vdash\ p'\cdot\dstep{\di{x}{i'}} : S \reach \di{x}{i'} \wedge p'\cdot\dstep{\di{x}{i'}} \le p\}
\end{equation*}
is a set of pairs of natural
numbers, on which the lexicographic ordering is well-founded. Thus it
has at least one smallest element, with associated path $q$ and declaration position $j$, 
and using lemma \ref{lem:pathmescomp}, $q$ is minimal for $<$ in $W$. 
We claim that $q$ is minimal among all reachable paths.
For suppose not; then there is a path $q' < q$ and an $i'$ such that  
\begin{equation*}
  \seeni,\seens \vdash\ q' : S \reach \di{x}{i'}. 
\end{equation*}
But since $q$ is in $W$ then $q \leq p$ and $<$ is transitive so $q' < p$ and then $q' \in W$,
contradicting the minimality of $q$ in $W$.
Therefore we have an element $q$ and position $j$ such that:
\begin{equation*}
  \seeni,\seens \vdash\ q : S \resolve \di{x}{j} \wedge q \le p 
\end{equation*}\qed
\end{proof}

\subsection{Correctness of resolution algorithm}

% \begin{lemma} \label{lemma:weakening}
%   $\forall\ \seeni\ \seens\ p\ S\ S'\ S''\ s.t. \seeni,\{S\}\cup\seens \vdash p  : S' \medge S'' \longrightarrow
%   \seeni,\seens \vdash p : S' \medge S''$  
% \end{lemma}
% \begin{proof}
%   Trivial, by induction on p, $B \notin \{S\}\cup\seens \Rightarrow B \notin \seens$. 
% \end{proof}

We now proceed to prove the correctness of the resolution algorithm. We first address some auxiliary lemmas that we will need in the proof.
First, the path uniquely encodes the sequence of scopes in the resolution: 
\begin{lemma}[Transition unique] \label{lemma:transuniq}
\begin{equation*}
\seeni,\seens \vdash s : S \edge S_1 \Longrightarrow \seeni,\seens \vdash s : S \edge S_2 \Longrightarrow S_1 = S_2
\end{equation*}
\end{lemma}
\begin{proof} Trivial by case analysis on $s$: import steps record the imported scope $S_y$ to which they step,
and parent steps go to the uniquely defined parent.\qed
\end{proof}

Then we state that we can cut a resolution path for the reachability relation, i.e., if a declaration is reachable from a scope $S$ then it is also reachable from any scope appearing in the resolution.

\begin{lemma}[Reachable tail] \label{lemma:tailreach} If a definition is reachable from a scope $S$ through scope $S'$ then it is also reachable from $S'$:\\
  \begin{equation*}
   \forall\ \seeni\ \seens\ s\ p\ S\ d,
    \begin{array}[t]{rl}
    \seeni,\{S\}\cup\seens & \vdash s : S \edge S' \Longrightarrow \\  
    \seeni,\seens & \vdash s \cdot p\cdot\dstep{\d{x}} : S \reach \d{x} \Longrightarrow \\
    & \seeni,\{S\}\cup\seens \vdash p\cdot\dstep{\d{x}} : S' \reach \d{x}    
    \end{array} 
  \end{equation*}
   
\end{lemma}
\begin{proof} If $s\cdot p$ is well formed then $p$ is also well formed. By inversion of:
  \begin{equation*}
    \seeni,\seens \vdash s \cdot p \cdot \dstep{\d{x}} : S \reach \d{x} 
  \end{equation*}
and using lemma \ref{lemma:transuniq}, we get:
\begin{enumerate}[leftmargin=15mm]
 \item $\seeni,\{S,S'\}\cup\seens \vdash p : S' \medge S_d $
 \item $S \notin \seens$
 \item $S' \notin \seens$
 \item $\d{x} \in \D{S_d}$
\end{enumerate}
Therefore, using $i)$, $iii)$ and $iv)$, we get by rule $(R')$:
\begin{equation*}
\seeni,\{S\}\cup\seens \vdash p\cdot\dstep{\d{x}} : S' \reach \d{x}   
\end{equation*}\qed
\end{proof}

Similarly, if a declaration is visible from a reference then it is visible all along the path.

\begin{lemma}[Visible tail] \label{lemma:tailres}
If a definition is visible from a scope $S$ through scope $S'$ then it is also visible from $S'$:\\
\begin{equation*}
\forall\ \seeni\ \seens\ s\ p\ S\ \di{x}{i}, 
\begin{array}[t]{rll}
\seeni,\{S\}\cup\seens & \vdash s : S \edge S' \Longrightarrow & (i) \\
\seeni,\seens & \vdash s \cdot p \cdot \dstep{\di{x}{i}} : S \resolve \di{x}{i} \Longrightarrow & (ii)\\ 
& \seeni,\{S\}\cup\seens \vdash p \cdot \dstep{\di{x}{i}} : S' \resolve \di{x}{i}  
\end{array}
\end{equation*}
\end{lemma}
\begin{proof} By inversion on $(ii)$
we get:
\begin{itemize}[leftmargin=15mm]
 \item[$(*)$\tab] $\seeni,\seens \vdash s \cdot p \cdot \dstep{\di{x}{i}} : S \reach \di{x}{i}$ 
 \item[$(**)$\tab] $\forall\ p'\ \di{x}{i'}\ s.t.\ \seeni,\seens \vdash p' : S \reach \di{x}{i'} \Rightarrow \neg p' < s\cdot p $ 
\end{itemize}
By lemma \ref{lemma:tailreach} and $(*)$ we have:
\begin{itemize}[leftmargin=15mm]
 \item[$(\lozenge)$\tab] $\seeni,\{S\}\cup\seens \vdash p \cdot \dstep{\di{x}{i}} : S' \reach \di{x}{i}$
\end{itemize}
Suppose $p$ is not minimal. Then there is a $p'$ such that:
\begin{itemize}[leftmargin=15mm]
 \item[$(Hr)$\tab] $\seeni,\{S\}\cup\seens \vdash p' \cdot \dstep{\di{x}{i'}} : S' \reach \di{x}{i'}$
 \item[$(<)$\tab] $ p' \cdot \dstep{\di{x}{i'}} < p \cdot \dstep{\di{x}{i}} $
\end{itemize}
If $s = \pstep$ then, since inversion on $(Hr)$ gives us $\WF(p')$, we obtain $\WF(s \cdot p')$.
Alternatively, if $s = \istep{\_}{\_}$ then, since inversion on $(*)$ gives us $\WF(s\cdot p)$, we know $p$
has the form $\istep{\_}{\_}^*$.
Then by (<), we must have that $p'$ also has the form $\istep{\_}{\_}^*$, so we again obtain $\WF(s \cdot p')$.
Therefore, by inversion of $(Hr)$ and using $(i)$ we have: \\
\begin{equation*}
\seeni,\seens \vdash s\cdot p'\cdot\dstep{\di{x}{i'}} : S \reach \di{x}{i'}  
\end{equation*}
which contradicts $(**)$.
Therefore $p$ is minimal and using $(\lozenge)$ we have:
\begin{equation*}
\seeni,\{S\}\cup\seens \vdash p \cdot \dstep{\di{x}{i}} : S' \resolve \di{x}{i}.
\end{equation*}\qed
\end{proof}

%To shorten notation, we write $\pathx{C}$ for $\paths{C}{\seeni}{\seens}{S}$ for each clause class $C$.
The elements in the $\pathx{C}$ sets are incomparable under the specificity ordering.

\begin{lemma}[Path set minima]\label{lemma:notinf}
\begin{equation*}
\forall\ C\ \seeni\ \seens\ S,\ \forall p, p' \in \paths{C}{\seeni}{\seens}{S},\ \neg p < p'.
\end{equation*}
\end{lemma}
\begin{proof} Assume for contradiction that $p < p' (*)$. Proceed by case analysis on $C$:
  \begin{itemize}
   \item Case $I$: by definition of $\pathx{I}$:
    \begin{enumerate}[leftmargin=15mm]
     \item $p = \istep{\r{y}}{\ds{y}{S_y}}\cdot \bar{p}$ 
     \item $p' = \istep{\r{z}}{\ds{z}{S_z}}\cdot \bar{p}'$
     \item $\seeni,\{S\}\cup\seens \vdash \bar{p} : S_y \resolve \di{x}{i} $
     \item $\seeni,\{S\}\cup\seens \vdash \bar{p}' : S_z \resolve \di{x}{i'} $
    \end{enumerate}
    By inversion of $(*)$, $\istep{\r{y}}{\ds{y}{S_y}} = \istep{\r{z}}{\ds{z}{S_z}}$ and hence $S_y = S_z$, so we cannot have $\bar{p} < \bar{p'}$ since both $\bar{p}$ and $\bar{p}'$ are minimal.
    \item Case $P$: Proof is similar to case $I$: both $\bar{p}$ and $\bar{p}'$ are resolutions from $\P{S}$.
    \item Cases $L,V,D$: trivial.\qed
  \end{itemize}
\end{proof}


\newcommand{\Case}[1]{\textbf{Case #1}:}
\renewcommand{\theenumi}{\arabic{enumi}}
\renewcommand{\labelenumi}{\theenumi)}


We now proceed to prove that the different functions of the algorithm compute definitions corresponding to paths in the $\pathx{C}$ sets.
\begin{lemma}
[Lemma~\ref{lemma:pathset-alg}]
For each class $C \in \{V,L,D,I,P\}$: 
\begin{equation*}
\forall\ \seeni\ \seens\ S, \Env{c}{\seeni}{\seens}{S} = \defsof{\paths{C}{\seeni}{\seens}{S}}
\end{equation*}
\end{lemma}
\begin{proof}
The proof is by three nested inductions,
the outer one on $\seeni$ (or, more strictly, on the size of \mbox{$|\R{\G} \setminus \seeni|$}, the references
\emph{not} in $\seeni$) and the second one on $\seens$ (more strictly, the size of \mbox{$|\S{\G} \setminus \seens|$}, 
the scopes \emph{not} in $\seens$) and the third one on the class $C$, with the order $V > L > D,I,P$.
We then pick an $\d{x}$, and proceed according to the class $C$:\smallskip

\noindent \Case{D} \\
$\d{x} \in \Env{D}{\seeni}{\seens}{S} \iff$ (by definition of $\Envu{D}$) \\
$\d{x} \in \D{S} \wedge S \not\in \seens \iff$ (by rule $(R')$)\\
$\seeni,\seens\vdash \dstep{\d{x}} : S \reach \d{x} \iff$ (by definition of $\Delta$ and $\pathx{D}$)\\
$\d{x} \in \defsof{\paths{D}{\seeni}{\seens}{S}}$\\

\noindent \Case{P}\\
$\d{x} \in \Env{P}{\seeni}{\seens}{S} \iff$ (by definition of $\Envu{P}$) \\
$\d{x} \in \Env{V}{\seeni}{\{S\} \cup \seens}{\P{S}} \wedge S \not\in \seens \iff$ (by induction on $\seens$)\\
$\d{x} \in \defsof{\paths{V}{\seeni}{\{S\} \cup \seens}{\P{S}}} \wedge S \not\in \seens \iff$ (by definition of $\Delta$, $\pathx{V}$)\\
% really need an additional little lemma here 
$\exists p\ s.t.\ \seeni,\{S\}\cup\seens \vdash p \cdot \dstep{\d{x}} : \P{S} \resolve \d{x} \wedge S \notin \seens$\\ 
\tab $\iff$ ($\Longrightarrow$ by rule $(P)$  and rules $(V',R')$; $\Longleftarrow$ by weakening)\\
$\exists p\ s.t.\ \seeni,\{S\}\cup\seens \vdash p \cdot \dstep{\d{x}} : \P{S} \resolve \d{x} \wedge S \notin \seens \wedge \\
\tab\tab\tab\seeni \vdash \pstep : S \edge \P{S} \wedge \P{S} \notin \{S\}\cup \seens$\\ 
\tab $\iff$ (by rules $(R')$ and $(T')$)\\
$\exists p\ s.t.\ \seeni,\{S\}\cup\seens \vdash p \cdot \dstep{\d{x}} : \P{S} \resolve \d{x} \wedge \seeni,\seens \vdash \pstep\cdot p\cdot \dstep{\d{x}} : S \reach \d{x}$ \\
\tab$ \iff$ (by definition of $\Delta$ and$\pathx{P}$)\\
% and again here
$ \d{x} \in \defsof{\paths{P}{\seeni}{\seens}{S}}$\\

\newcommand{\ttab}{\tab\tab\tab}
\noindent \Case{I}\\
$\d{x} \in \Env{I}{\seeni}{\seens}{S} \iff$ (definition of $\Envu{I}$) \\
$\d{x} \in \bigcup \left\{ \Env{L}{\seeni}{\{S\}\cup\seens}{S_y} \mid \r{y} \in \I{S}\setminus \seeni \wedge \ds{y}{S_y} \in \Res{\seeni}{\r{y}}\right\} \wedge S \not\in \seens$\\
\tab $ \iff$ (by unfolding of $\Res{\seeni}{\r{y}}$ and induction on $\seeni$) \\
$\d{x} \in \bigcup \left\{ \Env{L}{\seeni}{\{S\}\cup\seens}{S_y} \left| 
  \begin{array}{l}
    \r{y} \in \I{S}\setminus \seeni  \wedge\\
    \exists S', \r{y} \in \R{S'} \wedge\\
    \tab \ds{y}{S_y} \in \defsof{\paths{V}{\{\r{y}\}\cup\seeni}{\emptyset}{S'}} 
  \end{array}\right\}\right. \wedge S \not\in \seens$\\
\tab $\iff$ (by definition of $\pathx{V}$ and rule $(X')$)\\
$\d{x} \in \bigcup \left\{ \Env{L}{\seeni}{\{S\}\cup\seens}{S_y} \left|  
  \begin{array}{l}
   \r{y} \in \I{S}\setminus\seeni \wedge \\
   \exists q\ \ds{y}{S_y}\ s.t.\ \seeni \vdash q: \r{y} \resolve \ds{y}{S_y}
\end{array}\right\}\right. \wedge S \not\in \seens$ \\
\tab $\iff$ (by $L$ case of induction on $\seens$) \\
$\exists \r{y}\ q\ \ds{y}{S_y}\ p\ s.t.\\
\ttab \r{y} \in \I{S}\setminus\seeni \wedge \seeni \vdash q:\r{y} \resolve \ds{y}{S_y} \wedge\\
\ttab p \cdot D(\d{x}) \in \paths{L}{\seeni}{\{S\}\cup\seens}{S_y} \wedge S \not\in \seens$\\
\tab$\iff$ (by definition of $\pathx{L}$)\\
$\exists \r{y}\ q\ \ds{y}{S_y}\ p\ s.t.\ \r{y} \in \I{S}\setminus\seeni \wedge \\
\ttab \seeni \vdash q:\r{y} \resolve \ds{y}{S_y} \wedge \\
\ttab \seeni,{\{S\}\cup\seens} \vdash p \cdot \dstep{\d{x}} : S_y \resolve \d{x} \wedge \\
\ttab p \in \istep{\_}{\_}^* \wedge\\
\ttab S \not\in \seens$\\
\tab$\iff$ (by rule $(I)$)\\
$\exists \r{y}\ \ds{y}{S_y}\ p,\\ 
\ttab \seeni \vdash \istep{\r{y}}{\ds{y}{S_y}} : S \edge S_y\\
\ttab \seeni,{\{S\}\cup\seens} \vdash p \cdot \dstep{\d{x}} : S_y \resolve \d{x} \wedge \\
\ttab p \in \istep{\_}{\_}^* \wedge\\
\ttab S \not\in \seens$\\
\tab $\iff$ ($\Longrightarrow$ by $(R')$; $\Longleftarrow$ by weakening)\\
$\exists\ S_x\ \r{y}\ \ds{y}{S_y}\ p,\\ 
\ttab \seeni \vdash \istep{\r{y}}{\ds{y}{S_y}} : S \edge S_y\\
\ttab S_y \notin {\{S\}\cup\seens}\\
\ttab \seeni,{\{S,S_y\}\cup\seens} \vdash p  : S_y \medge S_x\\
\ttab \d{x} \in S_x \wedge\\
\ttab \seeni,{\{S\}\cup\seens} \vdash p \cdot \dstep{\d{x}} : S_y \resolve \d{x} \wedge \\
\ttab p \in \istep{\_}{\_}^* \wedge\\
\ttab S \not\in \seens$\\
\tab $\iff$ (by $(T')$, $(I)$ and $\WF$ definition)\\
$\exists\ S_x\ \r{y}\ \ds{y}{S_y}\ p,\\ 
\ttab \seeni,{\{S\}\cup\seens} \vdash \istep{\r{y}}{\ds{y}{S_y}}\cdot p : S \medge S_x\\
\ttab \d{x} \in S_x \wedge\\
\ttab \seeni,{\{S\}\cup\seens} \vdash p \cdot \dstep{\d{x}} : S_y \resolve \d{x} \wedge \\
\ttab \WF(\istep{\r{y}}{\ds{y}{S_y}}\cdot p \cdot \dstep{\d{x}}) \wedge\\
\ttab S \not\in \seens$\\
\tab $\iff$ (by $(R')$)\\
$\exists\ \r{y}\ \ds{y}{S_y}\ p,\\ 
\ttab \seeni,{\seens} \vdash \istep{\r{y}}{\ds{y}{S_y}}\cdot p \cdot \dstep{\d{x}}: S \reach \d{x} \wedge\\
\ttab \seeni,{\{S\}\cup\seens} \vdash p \cdot \dstep{\d{x}} : S_y \resolve \d{x} $\\
\tab $\iff$ (by definition of $\pathx{I}$)\\
$\exists\ \r{y}\ \ds{y}{S_y}\ p,\\
\ttab \istep{\r{y}}{\ds{y}{S_y}}\cdot p \cdot \dstep{\d{x}} \in \paths{I}{\seeni}{\seens}{S}$\\
\tab $\iff$ (by definition of $\Delta$)\\
$\d{x} \in \defsof{\paths{I}{\seeni}{\seens}{S}}$\\

\noindent \Case{L} 
We split the two directions:

\Case{L.($\Rightarrow$)} \\
$\d{x} \in \Env{L}{\seeni}{\seens}{S} \Rightarrow$ (by definition of $\Envu{L}$) \\
$\d{x} \in \Env{D}{\seeni}{\seens}{S} \hiding \Env{I}{\seeni}{\seens}{S}$\\
By case on the set $\d{x}$ comes from:
\begin{enumerate}
 \item If $\d{x} \in \Env{D}{\seeni}{\seens}{S}$ \\
 then by the induction on the class (since $D < L$) we get:\\
 $\d{x} \in \defsof{\paths{D}{\seeni}{\seens}{S}} \Rightarrow$ (by definition of $\Delta$ and $\pathx{D}$)\\
 $\seeni,\seens \vdash \dstep{\d{x}} : S \reach \d{x} \Rightarrow$ ($\dstep{\d{x}}$ is always minimal)\\ 
 $\seeni,\seens \vdash \dstep{\d{x}} : S \resolve \d{x} \Rightarrow$ (since $\dstep{\d{x}} \in \istep{\_}{\_}^*\cdot \dstep{\_}$)\\
 $\dstep{\d{x}} \in \paths{L}{\seeni}{\seens}{S}\Rightarrow$ (by definition)
 $\d{x} \in \defsof{\paths{L}{\seeni}{\seens}{S}}$
 \smallskip

\item If $\forall\ i,\ \di{x}{i} \notin \Env{D}{\seeni}{\seens}{S}$ and $\d{x} \in \Env{I}{\seeni}{\seens}{S}$
 then by the induction on class (since $I < L$) we get that:
 $\d{x} \in \defsof{\paths{I}{\seeni}{\seens}{S}}$\\
By definition of $\pathx{I}$ there exist $p\ p'\ \r{y}\ \ds{y}{S'}$, such that $p = \istep{\r{y}}{\ds{y}{S'}}  \cdot p'$ and:
\begin{itemize}[leftmargin=15mm]
 \item[$(*)$] $ \seeni,\seens \vdash p : S \reach \d{x}$
 \item[$(**)$] $ \seeni,\{S\} \cup \seens \vdash p' : S' \resolve \d{x}$
\end{itemize}
From $(*)$ we have $\WF(\istep{\r{y}}{\ds{y}{S'}}  \cdot p')$, which trivially implies that
 $p \in \istep{\_}{\_}^*\cdot \dstep{\_}$.\\
Thus, to prove that $\d{x} \in \defsof{\paths{L}{\seeni}{\seens}{S}}$ it is sufficient to prove that $ \seeni,\seens \vdash p : S \resolve \d{x}$ and given $(*)$ we need only 
prove that $p$ is minimal:\\
Assume for contradiction that there is a path $\bar{p}$ such that:
\begin{itemize}[leftmargin=15mm]
 \item[$ (\diamond)$] $ \seeni,\seens \vdash \bar{p} : S \reach \di{x}{i}$
\end{itemize}
and $\bar{p} < \istep{\r{y}}{\ds{y}{S'}}\cdot p'$. Then we have 2 cases:
\begin{itemize}[leftmargin=10mm]
 \item $\bar{p} = \dstep{\di{x}{i}}$:\\
  then by definition of $\pathx{D}$ it contains a resolution path for $x$ and by induction hypothesis on the class ($D < L$) there is a definition $\di{x}{i} \in \Env{D}{\seeni}{\seens}{S}$: contradiction.
 \item $\bar{p} = \istep{\r{y}}{\ds{y}{S'}}\cdot \bar{p}' \wedge \bar{p}' < p'$:\\
  then by $(\diamond)$ and lemma \ref{lemma:tailreach}, we have that:\\
  \tab $\seeni,\{S\} \cup \seens \vdash \bar{p}' : S' \reach \di{x}{i}$ \\
  which contradicts $(**)$ since $\bar{p}' < p'$.\medskip
\end{itemize}
\end{enumerate}

\Case{L.($\Leftarrow$)} 
Assume $\d{x} \in \defsof{\paths{L}{\seeni}{\seens}{S}}$. Then there is a path $p$ such that:
\begin{itemize}[leftmargin=15mm]
 \item[$(*)$] $\seeni,\seens \vdash p : S \resolve \d{x}$
 \item[$(**)$] $p \in \istep{\_}{\_}^*\cdot  \dstep{\_}$
\end{itemize}
By rule $(V')$ we have that:
\begin{itemize}[leftmargin=15mm]
 \item[$(i)$] $\seeni,\seens \vdash p : S \reach \d{x}$
 \item[$(ii)$] $p$ is a minimal path for $x$ under $\seeni,\seens$.
\end{itemize}
By case analysis on the first step of $p$: 
\begin{enumerate}
  \item Case $p =  \dstep{\d{x}}$: \\
   $\d{x}$ is trivially in $\Env{D}{\seeni}{\seens}{S}$ and thus in $\Env{L}{\seeni}{\seens}{S}$.
 \item Case $p = \istep{\r{y}}{\ds{y}{S'}}\cdot p'$:\\
 By lemma \ref{lemma:tailres} and $(*)$ we have:\\
\tab $\seeni,\{S\} \cup \seens \vdash p' : S' \resolve \d{x}$ \\
Combining this with $(i)$  we have $p \in \paths{I}{\seeni}{\seens}{S}$ 
so by induction on the class, $\d{x} \in \Env{I}{\seeni}{\seens}{S}$.
Moreover, there is no $i$ with $\di{x}{i} \in \Env{D}{\seeni}{\seens}{S}$: if there were, 
we would have a path $q = \dstep{\di{x}{i}}$ with $q < p$ and $\seeni,\seens \vdash q : S \reach \di{x}{i}$, contradicting $(ii)$.
So $\d{x} \in\Env{L}{\seeni}{\seens}{S}$.
\end{enumerate}

\noindent \Case{V}
We split the two directions:

\Case{V.($\Rightarrow$)}\\ 
$\d{x} \in \Env{V}{\seeni}{\seens}{S} \Rightarrow$ (by definition of $\Envu{L}$) \\
$\d{x} \in \Env{L}{\seeni}{\seens}{S} \hiding \Env{P}{\seeni}{\seens}{S}$\\
By case on the set $\d{x}$ comes from:
\begin{enumerate}
 \item If $\d{x} \in \Env{L}{\seeni}{\seens}{S}$ \\
 then by induction on the class (since $L < V$) we get:\\
 $\d{x} \in \defsof{\paths{L}{\seeni}{\seens}{S}}$\\
By definition of $\pathx{L}$ there is a path $p$ in $\istep{\_}{\_}^*\cdot \dstep{\_}$ such that:\\
 $\seeni,\seens \vdash p : S \resolve \d{x}$\\
And we have the conclusion:\\
 $\d{x} \in \defsof{\paths{V}{\seeni}{\seens}{S}}$

\item If $\forall\ i,\ \di{x}{i} \notin \Env{L}{\seeni}{\seens}{S}$ and $\d{x} \in \Env{P}{\seeni}{\seens}{S}$ 
then by the induction on the class (since $P < V$) we get that:
 $\d{x} \in \defsof{\paths{P}{\seeni}{\seens}{S}}$.\\
By definition of $\pathx{P}$ there exists $p'$ such that $p = \pstep \cdot p'$ and:
\begin{itemize}[leftmargin=15mm]
 \item[$(*)$] $ \seeni,\seens \vdash p : S \reach \d{x}$
 \item[$(**)$] $ \seeni,\{S\} \cup \seens \vdash p' : \P{S} \resolve \d{x}$
\end{itemize}
Given $(*)$, to prove that $\d{x} \in \defsof{\paths{V}{\seeni}{\seens}{S}}$ it is sufficient to prove that $p$ is minimal:\\
Assume for contradiction that there is a path $\bar{p}$ such that:
\begin{itemize}[leftmargin=15mm]
 \item[$ (\diamond)$] $ \seeni,\seens \vdash \bar{p} : S \reach \di{x}{i}$
\end{itemize}
and $\bar{p} < \pstep \cdot p'$. 
Then we have 2 cases:
\begin{itemize}[leftmargin=10mm]
 \item $\bar{p} \in \istep{\_}{\_}^*\cdot \dstep{\_}$:\\
  Then by $ (\diamond)$ and Lemma \ref{lemma:shadowexists} there is a path $\bar{p}'$ and a definition $\di{x}{i'}$ such that:
  \begin{itemize}
   \item[$(i)$]  $ \seeni,\seens \vdash \bar{p}' : S \resolve \di{x}{i'}$ 
   \item[$(ii)$]$\bar{p}' \leq \bar{p}$
  \end{itemize}
  From $(ii)$ we know  $\bar{p}' \in \istep{\_}{\_}^*\cdot \dstep{\_}$ and thus $\bar{p}' \in \paths{L}{\seeni}{\seens}{S}$.
By induction ($L < V$) we get that $\di{x}{i'} \in \Env{L}{\seeni}{\seens}{S}$: contradiction.

 \item $\bar{p} = \pstep\cdot \bar{p}' \wedge \bar{p}' < p'$:\\
  then by $(\diamond)$ and lemma \ref{lemma:tailreach}, we have that:\\
  \tab $\seeni,\{S\} \cup \seens \vdash \bar{p}' : \P{S} \reach \di{x}{i'}$ \\
  which contradicts $(**)$ since $\bar{p}' < p'$.\medskip
\end{itemize}
\end{enumerate}

\Case{V.($\Leftarrow$)} Assume $\d{x} \in \defsof{\paths{V}{\seeni}{\seens}{S}}$ then there is a path $p$ such that:
\begin{itemize}[leftmargin=15mm]
 \item[$(*)$] $\seeni,\seens \vdash p : S \resolve \d{x}$
\end{itemize}
By rule $(V')$ we have that:
\begin{itemize}[leftmargin=15mm]
 \item[$(i)$] $\seeni,\seens \vdash p : S \reach \d{x}$
 \item[$(ii)$] $p$ is a minimal path for $x$ under $\seeni,\seens$.
\end{itemize}
By case analysis on the first step on $p$: 
\begin{enumerate}
 \item Case $p \in \istep{\_}{\_}^*\cdot \dstep{\_}$: then $\d{x}$ is trivially in $\defsof{\paths{L}{\seeni}{\seens}{S}}$ and 
thus in $\Env{L}{\seeni}{\seens}{S}$ by induction on class and thus in $\Env{V}{\seeni}{\seens}{S}$.

 \item Case $p = \pstep \cdot p'$:\\
 By lemma \ref{lemma:tailres} and $(*)$ we have:\\
\tab $\seeni,\{S\} \cup \seens \vdash p' : \P{S} \resolve \d{x}$ \\
From this and $(i)$ we know $p \in \paths{P}{\seeni}{\seens}{S}$ and by induction on the class, $\d{x} \in \Env{P}{\seeni}{\seens}{S}$.
To show $\d{x} \in \Env{V}{\seeni}{\seens}{S}$, it suffices to show that
$\Env{L}{\seeni}{\seens}{S}$ cannot contain a definition $\di{x}{i}$. 
Suppose for contradiction that it does. 
Then by induction on class there is a resolution path $\bar{p}$ for $\di{x}{i}$ in 
$\paths{L}{\seeni}{\seens}{S}$.  By definition of $\pathx{L}$, $\bar{p}$ has the form
$\istep{\_}{\_}^*\cdot \dstep{\_}$, so $\bar{p} < p$, contradicting $(ii)$.
\end{enumerate}

This last case concludes the proof. \qed
\end{proof}

\subsection{\a-equivalence}\label{subsection:aeqproof}

We prove Lemma \ref{lemma:freevarclass} about the absence of declaration in the equivalence class of a free variable:
\begin{lemma}[Lemma \ref{lemma:freevarclass}] The equivalence class of a free variable can not contain any other declaration:
\begin{equation*}
\forall \di{x}{i}, i \seq{\mtt{P}} \bar{x} \Longrightarrow i = \bar{x}
\end{equation*}
\end{lemma}
\begin{proof}
  First, since $\top$ is the only path to the free variable definition we have:\\ 
  \tab$\forall\ \ri{x}{i}\ p,\ (\seeni \vdash p : \ri{x}{i} \resolve \di{x}{\bar{x}}) \Longrightarrow p = \top$\\
  Then since $\top$ is less specific than any other path and using the $(V)$ rule we have: 
  \tab$\forall\ \ri{x}{i},\ (\seeni \vdash \top : \ri{x}{i} \resolve \di{x}{\bar{x}}) \Longrightarrow \forall\ p\ i',\ \seeni \vdash p : \ri{x}{i} \resolve \di{x}{i'} \Longrightarrow i' = \bar{x}$\\
  So if there is resolution to the free variable declaration, then it is the only possible resolution for the reference. Finally, we prove by induction on the equivalence relation $i \seq{\mtt{P}} i'$ that any term
equivalent to $\bar{x}$ resolves to $\bar{x}$ or is $\bar{x}$ itself:
  \begin{multline*}
    \forall\ i\ j,\ i \seq{\mtt{P}} j \Longrightarrow\\
    \begin{array}{rl}
    ((j = \bar{x}\ \vee\ \seeni \vdash \top : \ri{x}{j} \resolve \di{x}{\bar{x}}) \Rightarrow (i = \bar{x}\ \vee\ \seeni \vdash \top : \ri{x}{i} \resolve \di{x}{\bar{x}})) & \wedge \\
    ((i = \bar{x}\ \vee\ \seeni \vdash \top : \ri{x}{i} \resolve \di{x}{\bar{x}}) \Rightarrow (j = \bar{x}\ \vee\ \seeni \vdash \top : \ri{x}{j} \resolve \di{x}{\bar{x}}))
    \end{array}
  \end{multline*}
Instantiating $j$ with $\bar{x}$ finishes the proof.\qed
\end{proof}
\endinput


\endinput

Let us now prove an equivalent definition of $\paths{V}{\seeni}{\seens}{S}$ stating that the paths in $\paths{V}{\seeni}{\seens}{S}$ are exactly
the resolution paths from S:

\begin{lemma}[Lemma~\ref{lemma:distr}]
  \begin{multline*}
\forall\ \seeni\ \seens\ S\ p\ \d{x}, p \cdot \dstep{\d{x}}\in\paths{V}{\seeni}{\seens}{S} \iff \seeni,\seens \vdash p \cdot \dstep{\d{x}}: S \resolve \d{x}    
  \end{multline*}
\end{lemma}

Proof: We fix $\seeni$, $\seens$ and proceed by case analysis on the first step of $p$.\medskip\\
($\Rightarrow$) Suppose $p \cdot \dstep{\d{x}}\in\pathx{V}$. By definition of $\pathx{V}$ and $\pathx{L}$
sets we have 
\begin{equation*}
  p \cdot \dstep{\d{x}}\in \visible(\visible(\pathx{D} \cup \pathx{I}) \cup \pathx{P}) 
\end{equation*}
Depending on the first step of $p$, we know the set $\pathx{D}$, $\pathx{I}$ or $\pathx{P}$ which $p$ comes from:\medskip\\

\noindent Case $p = []$:\\
since $\dstep{\d{x}}\in \pathx{D}$, it is trivial by definition of $\pathx{D}$ since $\dstep{\_}$ is always minimal.\medskip

\noindent Case $p \cdot \dstep{\di{x}{i}} = \istep{\r{y}}{\ds{y}{S'}}\cdot p'$: \\
$p \in \pathx{I}$ and $\pathx{D}$ does not contain a resolution for $x$ (otherwise $p\cdot \dstep{\di{x}{i}}$ is not in the $\visible$).
By definition of $\pathx{I}$ we have:
\begin{itemize}[leftmargin=15mm]
 \item[$(*)$]  $\seeni,\seens \vdash p\cdot \dstep{\di{x}{i}} : S \reach \di{x}{i}  $
 \item[$(**)$]  $\seeni,\{S\}\cup\seens \vdash p' : S \resolve \di{x}{i}  $
\end{itemize}
Given $(*)$, to get the conclusion we need to prove that $p\cdot \dstep{\di{x}{i}}$ is minimal. Assume there is a path $\bar{p}$ such that:
\begin{itemize}[leftmargin=15mm]
 \item[$ (\diamond)$] $ \seeni,\seens \vdash \bar{p} : S \reach \di{x}{i'}$
\end{itemize}
and $\bar{p} < \istep{\r{y}}{\ds{y}{S'}}\cdot p'$. Then we have 2 cases:
\begin{itemize}[leftmargin=10mm]
 \item $\bar{p} = \dstep{\di{x}{i'}}$:\\
  then by definition of $\pathx{D}$ it contains a resolution for $x$: contradiction
 \item $\bar{p} = \istep{\r{y}}{\ds{y}{S'}}\cdot \bar{p}' \wedge \bar{p}' < p'$:\\
  then by $(\diamond)$ and lemma \ref{lemma:tailreach}, we have that:\\
  \tab $\seeni,\{S\} \cup \seens \vdash \bar{p}' : S' \reach \di{x}{i'}$ \\
  which contradicts $(**)$ since $\bar{p}' < p'$.\medskip
\end{itemize}

\noindent Case $p \cdot \dstep{\di{x}{i}} = \pstep\cdot p'$:\\ 
$p \in \pathx{P}$ and $\pathx{D} \cup \pathx{I}$ does not contain a resolution for $x$ (if not $p$ is not in the $\visible$).
By definition of $\pathx{P}$ we have:
\begin{itemize}[leftmargin=15mm]
 \item[$(*)$] $ \seeni,\seens \vdash p\cdot \dstep{\di{x}{i}} : S \reach \di{x}{i}$
 \item[$(**)$] $ \seeni,\{S\} \cup \seens \vdash p' : \P{S} \resolve \di{x}{i}$
\end{itemize}
Given $(*)$, to get the conclusion we need to prove that $p\cdot \dstep{\di{x}{i}}$ is minimal. Assume there is a path $\bar{p}$ such that:
\begin{itemize}[leftmargin=15mm]
 \item[$(\diamond)$] $ \seeni,\seens \vdash \bar{p} : S \reach \di{x}{i'}$
\end{itemize}
and $\bar{p} < p\cdot \dstep{\di{x}{i}}$ then we have 3 cases:
\begin{itemize}[leftmargin=10mm]
 \item $\bar{p} = \dstep{\di{x}{i'}}$:\\
  then by definition of $\pathx{D}$ it contains a resolution for $x$: contradiction
 \item $\bar{p} = \istep{\r{y}}{\ds{y}{S'}}\cdot \bar{p}'$:\\
  then by $(\diamond)$ and lemma \ref{lemma:tailreach}, we have that:\\
  \tab $\seeni,\{S\} \cup \seens \vdash \bar{p}' : S' \reach \di{x}{i'}$\\
  Using lemma \ref{lemma:shadowexists}, we get a path $\bar{p}''$ and a definition $\di{x}{i''}$ such that:
  \begin{itemize}[leftmargin=15mm]
   \item[$(i)$] $ \seeni,\{S\} \cup \seens \vdash \bar{p}'' : S' \resolve \di{x}{i''}$
   \item[$(ii)$] $\bar{p}'' < \bar{p}'$
  \end{itemize}
  Since $\bar{p}$ and $\bar{p}''$ are well-formed and using $(ii)$ then we have that:\\
  \tab $\bar{p}'' \in \istep{\_}{\_}^*\cdot \dstep$\\
  and thus:\\
  \tab $WF(\istep{\r{y}}{\ds{y}{S'}}\cdot \bar{p}'')$\\
  Therefore using $(i)$ and $(\diamond)$ we get:
  \begin{itemize}[leftmargin=15mm]
   \item[(iii)] $\seeni,\seens \vdash \istep{\r{y}}{\ds{y}{S'}}\cdot \bar{p}'' : S \reach \di{x}{i''}$
  \end{itemize}
  Then by $(i)$ and $(iii)$, $\bar{p}''$ is a resolution for $x$ in $\pathx{I}$ which is a contradiction.
 \item $\bar{p} = \pstep \cdot \bar{p}' \wedge \bar{p}' < p'$:\\
 then by $(\diamond)$ and lemma \ref{lemma:tailreach}, we have that:\\
 \tab $\seeni,\{S\} \cup \seens \vdash \bar{p}' : \P{S} \reach \di{x}{i'}$ 
 which contradicts $(**)$ since $\bar{p}' < p'$ 
  \bigskip
\end{itemize}

\noindent ($\Leftarrow$) Suppose that:
\begin{itemize}[leftmargin=15mm]
 \item[$(*)$] $\seeni,\seens \vdash p \cdot \dstep{\d{x}}: S \resolve \d{x}$
\end{itemize}
By rule $(V')$ we have that:
\begin{itemize}[leftmargin=15mm]
 \item[$(i)$] $\seeni,\seens \vdash p \cdot \dstep{\d{x}} : S \reach \d{x}$
 \item[$(ii)$] $p\cdot \dstep{\d{x}}$ is a minimal path for $x$ under $\seeni,\seens$.
\end{itemize}
By case analysis on the first step of $p$: \medskip

\noindent Case $p = []$: \\
then $p\cdot \dstep{\d{x}}$ is trivially in $\pathx{D}$ and thus in the $\visible$.\medskip

\noindent Case $p\cdot \dstep{\d{x}} = \istep{\r{y}}{\ds{y}{S'}}\cdot p'$:\
 by lemma \ref{lemma:tailres} and $(*)$ we have:\\
\tab $\seeni,\{S\} \cup \seens \vdash p' : S' \resolve \d{x}$ \\
So, using $(i)$, we have $p \cdot \dstep{\d{x}} \in \pathx{I}$ and by lemma \ref{lemma:notinf} $p\cdot \dstep{\d{x}} \in \visible(\pathx{I})$.\\
If $\pathx{D}$ contains a path for $x$ then it is trivially smaller than $p\cdot \dstep{\d{x}}$ and thus $p\cdot \dstep{\d{x}}$ can not be minimal which contradicts $(ii)$. So $\pathx{D}$ does not contain a path for $x$ and therefore:\\
\tab $p\cdot \dstep{\d{x}} \in \visible(\pathx{D} \cup \pathx{I})$.\\
Therefore, using lemma \ref{lemma:min} for $\pathx{P}$, $p\cdot \dstep{\d{x}} \in \pathx{V}$.\medskip

\noindent Case $p\cdot \dstep{\d{x}} = \pstep\cdot p'$: \\
by lemma \ref{lemma:tailres} and $(*)$ we have:\\
\tab $\seeni,\{S\} \cup \seens \vdash p' : \P{S} \resolve \d{x}$\\
So, using $(i)$, we have $p \cdot \dstep{\d{x}} \in \pathx{P}$ and by lemma \ref{lemma:notinf} $p\cdot \dstep{\d{x}} \in \visible(\pathx{P})$.\\
If $\pathx{D}$ or $\pathx{I}$ contains a path for $x$ then it is trivially smaller than $p\cdot \dstep{\d{x}}$ and thus $p\cdot \dstep{\d{x}}$ can not be minimal which contradicts $(ii)$. 
Therefore, $\pathx{P}$, $p\cdot \dstep{\d{x}} \in \pathx{V})$.\qed \bigskip

 \begin{lemma}\label{lemma:pathl} Similar to lemma \ref{lemma:distr} for $\pathx{L}$:\\
$\forall\ \seeni\ \seens\ S\ p\ \d{x},
 p \cdot \dstep{\d{x}}\in\paths{L}{\seeni}{\seens}{S} \iff \seeni,\seens \vdash p \cdot \dstep{\d{x}}: S \resolve \d{x} \wedge p \in \istep{\_}{\_}^*$
 \end{lemma}
 \begin{proof} Fix $\seeni$,\ $\seens$,\ $S$,\ $p$ and $\d{x}$

\noindent ($\Rightarrow$) If $p \cdot \dstep{\d{x}} \in \paths{L}{\seeni}{\seens}{S}$ then $p \in \paths{V}{\seeni}{\seens}{S}$ since for all $p'$ in $\paths{P}{\seeni}{\seens}{S}$ we have $p \cdot \dstep{\d{x}} < p'$.
Then by lemma \ref{lemma:distr}, we have that:\\
\tab $\seeni,\seens \vdash p \cdot \dstep{\d{x}}: S \resolve \d{x}$\\
We now need to prove that $p \in \istep{\_}{\_}^*$, we have 2 cases:
\begin{itemize}
 \item if $p \cdot\dstep{\d{x}}\in \paths{D}{\seeni}{\seens}{S}$ then $p=[]$ and thus $p\in \istep{\_}{\_}^*$
 \item if $p \cdot\dstep{\d{x}}\in \paths{I}{\seeni}{\seens}{S}$ then:\\
\tab $\exists\ \r{y}\ \d{y}\ p'\ s.t.\ p = \istep{\r{y}}{\d{y}}\cdot p'$\\
and by inversion on:
\tab $\seeni,\seens \vdash \istep{\r{y}}{\d{y}} \cdot p' \cdot\dstep{\d{x}}  : S \reach \d{x}$
we have $WF(\istep{\r{y}}{\d{y}} \cdot p')$ and thus $\istep{\r{y}}{\d{y}} \cdot p' (= p) \in  \istep{\_}{\_}^*$.
\end{itemize}\medskip

\noindent ($\Leftarrow$) Assume:
\tab $\seeni,\seens \vdash p \cdot \dstep{\d{x}}: S \resolve \d{x} \wedge p \in \istep{\_}{\_}^*$
then by lemma \ref{lemma:distr}, $p \cdot \dstep{\d{x}} \in \paths{V}{\seeni}{\seens}{S}$. 
Since $p$ in $\istep{\_}{\_}^*$ then $p$ can not be in $\paths{P}{\seeni}{\seens}{S}$ so $p$ has to be in $\paths{L}{\seeni}{\seens}{S}$.\qed
 \end{proof}



%%% Local Variables: 
%%% mode: latex
%%% TeX-master: "../document"
%%% End: 

