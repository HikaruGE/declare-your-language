\section{Scope Graph Construction}
\sectionlabel{construction}


%% \newcommand{\semop}[1]{[\![ #1 ]\!]}
%% \newcommand{\sem}[2]{\semop{#1}(#2)}
%% % PCFM code for math mode
%% \newcommand{\semcode}[1]{\text{\pcfmcodefig{#1}}}
%% % Formula for regular interpretation translation
%% \newcommand{\semf}[4]{
%%  $\semop{#1}(#2) = #3$\\
%% \smallskip
%% \qquad$\text{\textsf{where}} \ #4 $
%% }
%% \newcommand{\semfs}[4]{
%% $\sems{#1}{#2} = #3$\\
%% \smallskip
%% \qquad$\text{\textsf{where}} \ #4 $
%% }
%% \newcommand{\semfp}[5]{
%% $\semp{#1}{#2}{#3} = #4$\\
%% \smallskip
%% \qquad$\text{\textsf{where}} \ #5 $
%% }

%% \newcommand{\semfline}[4]{
%%  $\semop{#1}(#2) = #3 \quad\text{\textsf{where}} \ #4 $
%% }

%% \newcommand{\semp}[3]{\semop{#1}^p(#2,#3)}
%% \newcommand{\sems}[2]{\semop{#1}^s(#2)}

%% \newcommand{\sappend}{\ \text{+=}\ }
%% \newcommand{\sdefine}{\ \text{:=}\ }

\newcommand{\sema}[3]{
\semop{#1}^{#2}_{#3}
}

%\newcommand{\snew}[1]{\text{\textsf{newscope}}\{\text{\textsf{par=}}#1\}}
\newcommand{\snew}[1]{\text{\textsf{new}}_{#1}}
\newcommand{\sret}[1]{\text{\textsf{ret}}(#1)}

\newcommand{\semeqn}[4]{
$\sema{#1}{#2}{#3}$ & $\sdefine$ & ${#4}$}

\newcommand{\slet}{\text{\textsf{let }}}
\renewcommand{\sin}{\text{\textsf{ in }}}

\newcommand{\myskip}{\\[4.5pt]}

\begin{figure}[tb]
%\fbox{
\begin{boxedminipage}{\hsize}
\begin{tabular}{rcl}
\semeqn{\semcode{ds}}{prog}{}
{
\slet S \sdefine \snew{\perp} \sin
\sema{\semcode{ds}}{recd}{S}
}
\myskip
\semeqn{\semcode{d ds}}{recd}{S}
{
\sema{\semcode{d}}{dec}{S}; 
\sema{\semcode{ds}}{recd}{S}
}
\\
\semeqn{}{recd}{S}{
()
}
\myskip
\semeqn{\semcode{module x}_i \semcode{ \{ds\}}}{dec}{S}
{
\slet S' \sdefine \snew{S} \sin 
\D{S} \sappend \dsi{x}{i}{S'}; 
\sema{\semcode{ds}}{recd}{S'}
}
\\
\semeqn{\semcode{import xs}}{dec}{S}
{
\sema{\semcode{xs}}{rqid}{S};
\sema{\semcode{xs}}{iqid}{S}
}
\\
\semeqn{\semcode{def x}_i\semcode{\ = e}}{dec}{S}
{
\D{S} \sappend \di{x}{i};
\sema{\semcode{e}}{exp}{S}
}
\myskip
\semeqn{\semcode{xs}}{exp}{S}
{
\sema{\semcode{xs}}{rqid}{S}
}
\\
\semeqn{\semcode{\(fun | fix\) x}_i\semcode{ \{e\}}}{exp}{S}
{
\slet S' \sdefine \snew{S} \sin
\D{S'} \sappend \di{x}{i};
\sema{\semcode{e}}{exp}{S'}
}
\\
\semeqn{\semcode{letrec bs in e}}{exp}{S}
{
\slet S' \sdefine \snew{S} \sin 
\sema{\semcode{bs}}{recb}{S'};
\sema{\semcode{e}}{exp}{S'}
}
\\
\semeqn{\semcode{letpar bs in e}}{exp}{S}
{
\slet S' \sdefine \snew{S} \sin 
\sema{\semcode{bs}}{parb}{(S,S')};
\sema{\semcode{e}}{exp}{S'}
}
\\
\semeqn{\semcode{let bs in e}}{exp}{S}
{
\slet S' \sdefine \sema{bs}{seqb}{S} \sin
\sema{\semcode{e}}{exp}{S'}
}
\\
\semeqn{\semcode{e}_1\ \semcode{e}_2}{exp}{S}
{
\sema{\semcode{e}_1}{exp}{S};
\sema{\semcode{e}_2}{exp}{S}
}
\\
\semeqn{\semcode{e}_1 \oplus \semcode{e}_2}{exp}{S}
{
\sema{\semcode{e}_1}{exp}{S};
\sema{\semcode{e}_2}{exp}{S}
}
\\
\semeqn{\semcode{n}}{exp}{S}
{
()
}
\myskip
\semeqn{\semcode{x}_i\semcode{.xs}}{rqid}{S}
{
\R{S} \sappend \ri{x}{i};
\slet S' \sdefine \snew{\perp} \sin 
\I{S'} \sappend \ri{x}{i};
\sema{xs}{rqid}{S'}
}
\\
\semeqn{\semcode{x}_i}{rqid}{S}
{
\R{S} \sappend \ri{x}{i}
}
\myskip
\semeqn{\semcode{x}_i\semcode{.xs}}{iqid}{S}
{
\sema{xs}{iqid}{S}
}
\\
\semeqn{\semcode{x}_i}{iqid}{S}
{
\I{S} \sappend \ri{x}{i}
}
\myskip
\semeqn{\semcode{x}_i\semcode{\ = e; bs}}{recb}{S}
{
\D{S} \sappend \di{x}{i};
\sema{\semcode{e}}{exp}{S};
\sema{\semcode{bs}}{recb}{S}
}
\\
\semeqn{}{recb}{S}
{
()
}
\myskip
\semeqn{\semcode{x}_i\semcode{\ = e; bs}}{parb}{(S,S')}
{
\D{S'} \sappend \di{x}{i};
\sema{\semcode{e}}{exp}{S};
\sema{\semcode{bs}}{parb}{(S,S')}
}
\\
\semeqn{}{parb}{(S,S')}
{
()
}
\myskip
\semeqn{\semcode{x}_i\semcode{\ = e; bs}}{seqb}{S}
{
\sema{\semcode{e}}{exp}{S};
\slet S' \sdefine \snew{S} \sin
\D{S'} \sappend \di{x}{i};
\sret{S'}
}
\\
\semeqn{}{seqb}{S}
{
\sret{S}
}
\end{tabular}
\end{boxedminipage}
%}
% }}\smallskip
\caption{Scope graph construction for LM via syntax-directed AST traversal.}
\label{fig:lm-scopegraph-construction}
\end{figure}


The preceding sections have illustrated scope graph construction
by means of examples corresponding to various language features. 
Of course, to apply our formalism in practice, one must be able
to construct scope graphs systematically. Ultimately, we would
like to be able to specify this process for arbitrary 
languages using a generic binding specification language such
as NaBL~\cite{KonatKWV12}, but that remains future work.
Here we illustrate systematic scope graph construction for arbitrary
programs in a \emph{specific} language, LM (\Figure{pcfm:grammar}), 
via straightforward syntax-directed traversal.  

Figure~\ref{fig:lm-scopegraph-construction} describes the
construction algorithm.
For clarity of presentation, the algorithm traverses the 
program's concrete syntax; a real implementation would
traverse the program's AST.
The algorithm is presented in an {\it ad hoc} imperative language, 
explained here.  The traversal is specified as a collection of 
(potentially) mutually recursive functions, one or more for
each syntactic class of LM. Each function $f$ is
defined by a set of clauses $\sema{\mbox{\it{pattern}}}{f}{args}$.
When $f$ is invoked on a term, the clause whose {\it{pattern}} matches
the term is executed.  Functions may also take additional arguments $args$. 
Each clause body consists of a sequence of
statements separated by semicolons. Functions can optionally return a value
using $\sret{}$.
The $\slet$ \hspace*{-0.6em} statement binds 
a metavariable in the remainder of the clause body.
An empty clause body is written $()$.

The algorithm is initiated by invoking $\sema{\_}{prog}{}$ on an entire LM
program. Its net effect is to produce a scope graph via a sequence of
imperative operations. 
The construct $\snew{P}$ creates a new scope $S$ with parent $P$ 
(or no parent if  $p = \perp$) and
empty sets $\D{S}$, $\R{S}$, and $\I{S}$. 
These sets are subsequently populated using the
$\sappend$ operator, which extends a set imperatively.
The program scope graph is simply the set of scopes that have been created
and populated when the traversal terminates.



