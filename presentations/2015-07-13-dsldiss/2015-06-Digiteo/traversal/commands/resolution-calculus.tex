% The sets that define a scope graph.
\newcommand{\G}{\mathcal{G}} 
\renewcommand{\S}[1]{\mathcal{S}(#1)}
\newcommand{\SG}{\mathcal{S}(\G)}
\newcommand{\D}[1]{\mathcal{D}(#1)}
%\newcommand{\DG}[2]{\mathcal{D}_{#1}(#2)}
\newcommand{\DG}{\mathcal{D}(\G )}
\newcommand{\R}[1]{\mathcal{R}(#1)}
%\newcommand{\RG}[2]{\mathcal{R}_{#1}(#2)}
\newcommand{\RG}{\mathcal{R}(\G)}
\newcommand{\I}[1]{\mathcal{I}(#1)}
\newcommand{\IG}[2]{\mathcal{I}_{#1}(#2)}
\newcommand{\IT}[1]{\mathcal{I}^t(#1)}
\renewcommand{\P}[1]{\mathcal{P}(#1)}
\newcommand{\PG}[2]{\mathcal{P}_{#1}(#2)}
\newcommand{\IS}[1]{\mathcal{IS}(#1)}

% References and declarations
%\renewcommand{\r}[1]{r_{#1}}   % reference
%\newcommand{\ri}[2]{r_{#1}^{\,#2}}  % reference with position
%\renewcommand{\d}[1]{d_{#1}} % declaration
%\newcommand{\di}[2]{d_{#1}^{\,#2}} % declaration with position
%\newcommand{\ds}[2]{d_{#1}\!\!:\!\!{#2}} % declaration with scope
%\newcommand{\dsi}[3]{d_{#1}^{\,#2}\!\!:\!\!{#3}} % declaration with position and scope
%\newcommand{\dsio}[3]{d_{#1}^{\,#2}{\mbox{\it[$:\!\!{#3}$]}}} % declaration with position and scope
% replace below with above to revert to old notation:
\newcommand{\goodsize}{\scriptsize}
\renewcommand{\r}[1]{{#1}^{\mbox{\goodsize\sf R}}}   % reference
\newcommand{\tr}[1]{{#1}^{\mbox{\goodsize\sf TR}}}   % transitive import reference
\newcommand{\er}[1]{{#1}^{\mbox{\goodsize\sf ER}}}   % export reference
\newcommand{\ri}[2]{{#1}^{\mbox{\goodsize\sf R}}_{#2}}  % reference with position
\renewcommand{\d}[1]{{#1}^{\mbox{\goodsize\sf D}}} % declaration
\newcommand{\di}[2]{{#1}^{\mbox{\goodsize\sf D}}_{#2}} % declaration with position
\newcommand{\ds}[2]{{#1}^{\mbox{\goodsize\sf D}}\!\!:\!\!{#2}} % declaration with scope
\newcommand{\dsi}[3]{{#1}^{\mbox{\goodsize\sf D}}_{#2}\!\!:\!\!{#3}} % declaration with position and scope

\newcommand{\scopeof}{\mathcal{S}c} % {\mathcal{S}c}
\newcommand{\rscopeof}[2]{\scopeof(#1) = #2}% {\scopeof(#1) = #2} % Scope of a position or a reference
\newcommand{\defscope}{\mathcal{DS}c}% {\scopeof(#1) = #2} % Scope of a p

\newcommand{\stable}[2]{#1 \uparrow {#2}}

% reference to reference in examples
\newcommand{\re}[2]{\lstinline{#1}$_{_{#2}}$}

\newcommand{\si}[1]{{#1}}
\newcommand{\pos}[1]{#1^{\textsf{P}}}


% Steps in paths:
\newcommand{\dstep}[1]{\mbox{\rm\bf{D}}(#1)}
\newcommand{\pstep}{\mbox{\rm\bf{P}}}
\newcommand{\istep}[2]{\mbox{\rm\bf{I}}(#1,#2)}
\newcommand{\sistep}[1]{\mbox{\rm\bf{I}}(#1)}

% Seen sets
\newcommand{\seeni}{\mathbb{I}}
\newcommand{\seens}{\mathbb{S}}

% Spacing stuff.
\newcommand{\premsep}{\hspace*{0.5cm}}
\newcommand{\tab}{\hspace*{0.3cm}}

\usepackage{relsize}

% Should be dead now, but temporarily used to define the next set of things
\newcommand{\lra}{\ \mathlarger{\mathlarger{\rightarrowtail}}\ }
\newcommand{\lrau}{\ \mathlarger{\mathlarger{\twoheadrightarrow}}\ }
\newcommand{\scopea}[2]{\longrightarrow_{#1}^{#2}}
\newcommand{\scopeau}[1]{\longrightarrow_{#1}}
\newcommand{\resolveau}{\longmapsto}
\newcommand{\resolvea}[1]{\longmapsto^{#1}}
\renewcommand{\scopea}[1]{\stackrel{#1}{\longrightarrow}}

% Resolution calculus relations
\newcommand{\edge}{\scopea{}}
\newcommand{\medge}{\lrau{}}  % multi-edge
\newcommand{\reach}{\lra}
\newcommand{\resolve}{\resolveau}

% Well-formedness
\newcommand{\WF}{\mbox{\it{WF}}}

% Environments and resolution
\newcommand{\Res}[2]{\mbox{\it{R}}[#1](#2)}
\newcommand{\Resx}[4]{\mbox{\it{R}}_{#1}[#2,#3](#4)}
\newcommand{\ResG}[2]{\mbox{\it{R}}_{#1}(#2)}
\newcommand{\ResGx}[3]{\mbox{\it{R}}_{#1}[#2](#3)}
\newcommand{\Env}[4]{\mbox{\it{Env}}_{#1}[#2,#3](#4)}
\newcommand{\Envu}[1]{\mbox{\it{Env}}_{#1}}

% Path sets
\newcommand{\pathx}[1]{\mathbb{P}_{#1}}
\newcommand{\paths}[4]{\mathbb{P}_{#1}[#2,#3](#4)}

\newcommand{\defsof}[1]{\Delta(#1)}
\newcommand{\defof}[1]{\delta(#1)}
% \newcommand{\bndof}[1]{\chi(#1)}

\newcommand{\hiding}{\triangleleft}

\newcommand{\visible}{\mbox{\it{visible}}}

% solver related commands
\newcommand{\exception}{\textit{\textbf{exception}}}
\newcommand{\where}{\textit{where }}
\newcommand{\whand}{\textit{and }}
\newcommand{\unify}[2]{\mathcal{U}(#1,#2)}
\newcommand{\unifystar}[1]{\mathcal{U}^{*}(#1)}


% Equivalence relation for alpha-eq
\renewcommand{\a}{$\alpha$}
\newcommand{\seq}[1]{\stackrel{#1}{\sim}}
\newcommand{\mtt}[1]{\text{\tt #1}}
\newcommand{\aeq}{\stackrel{\alpha}{\approx}}

%Type constructors
\newcommand{\TypeCons}{C_\mathcal{T}}

%Command for constraints
\newcommand{\teq}[2]{#1 \equiv #2}
\newcommand{\Cassoc}{{\rightsquigarrow}}
\newcommand{\Aassoc}{\leadsto}
\newcommand{\extends}{{\triangleleft}}
\newcommand{\msort}[1]{{\langle{#1}\rangle}}
\newcommand{\mlit}[1]{{\mbox{\normalfont\ttfamily #1}}}
\newcommand{\mul}[1]{{\overrightarrow{#1}}}
\newcommand{\ground}[1]{{\text{fv}({#1}) = \emptyset}}
\newcommand{\true}{\text{\sf True}}

% color coding
% palette is at http://en.wikibooks.org/wiki/LaTeX/Colors dvips 68 standard colors
\newcommand{\assumpcolor}[1]{\textcolor{blue}{#1}}
\newcommand{\subtypecolor}[1]{\textcolor{cyan}{#1}}
\newcommand{\typconstraintcolor}[1]{\textcolor{Mahogany}{#1}}
\newcommand{\resconstraintcolor}[1]{\textcolor{ForestGreen}{#1}}

% Constraints macros
%% Assumptions
\newcommand{\undef}{\perp}
\newcommand{\AParent}[2]{\assumpcolor{P({#1}):={#2}}} % {P({#1})\mbox{\it\ is }{#2}}
%\newcommand{\AScopeof}[2]{\assumpcolor{\scopeof({#1}):={#2}}} %{\scopeof({#1})\mbox{\it\ is }{#2}}
\newcommand{\AR}[2]{\assumpcolor{{#1} \in \R{#2}}}
\newcommand{\AD}[2]{\assumpcolor{{#1} \in \D{#2}}}
\newcommand{\ARImport}[2]{\assumpcolor{{#1} \in \I{#2}}}
\newcommand{\ASImport}[2]{\assumpcolor{{#1} \in \IS{#2}}}
\newcommand{\ASubtype}[2]{\subtypecolor{{#1} <: {#2}}}
\newcommand{\AAssoc}[2]{\assumpcolor{{#1} \Aassoc {#2}}}
\newcommand{\AScopeDef}[2]{\assumpcolor{{#1} :: {#2}}}

%% Constraints
\newcommand{\CTrue}{\text{\sf True}}  % don't use constraint color
\newcommand{\CResolve}[2]{\resconstraintcolor{{#1} \mapsto {#2}}}
\newcommand{\CTeq}[2]{\typconstraintcolor{\teq{#1}{#2}}}
\newcommand{\CAnd}[2]{{#1} \wedge {#2}}    % don't use constraint color 
\newcommand{\CAssoc}[2]{\resconstraintcolor{{#1} \Cassoc {#2}}}
\newcommand{\CSubtype}[2]{\typconstraintcolor{{#1} \preceq {#2}}}
\newcommand{\CType}[2]{\typconstraintcolor{{#1} : {#2}}}
\newcommand{\CLub}[3]{\typconstraintcolor{#1\mbox{\it\ is } #2 \sqcup #3}}

\newcommand{\lubsem}[2]{\sqcup_{#1} #2} 


%% an alias for now
\newcommand{\CEqtype}[2]{\CTeq{#1}{#2}}

\newcommand{\var}[1]{\mathcal{V}(#1)}
\newcommand{\lubred}[1]{\stackrel{#1}{\longrightarrow}}

% Old stuff now unused
%\newcommand{\Def}[1]{\mbox{\it D}(#1)}
%\newcommand{\Ref}[1]{\tpl{#1}}%{\hbox{\it R}(#1)}
%\newcommand{\Imp}[1]{\mbox{\it I}(#1)}
%\newcommand{\Impp}[1]{I(#1)} % for use as subscript
%\newcommand{\Lab}[1]{\hbox{(#1)}}

%\newcommand{\Lex}{Lex}
%\newcommand{\Loc}{Loc}

%\newcommand{\Res}[2]{\hbox{\it R}_{#2}(#1)}


%\newcommand{\nilscope}{\varnothing}
%\newcommand{\shadowset}{\lhd}
%\newcommand{\rootscope}{\top}
%\newcommand{\ntrans}{\hbox{NT}}
%\newcommand{\trans}{\hbox{T}}

%\newcommand{\impset}{\mathcal{I}}
%\newcommand{\scoset}{\mathcal{S}}

%\renewcommand{\d}{\delta}

%\newcommand{\longtwoheadrightarrow}{\mathrel{\longrightarrow\!\!\!\!\!\rightarrow}}
%\newcommand{\lra}[1]{\stackrel{#1}{\longtwoheadrightarrow}}

%%% Local Variables: 
%%% mode: latex
%%% TeX-master: "../document"
%%% End: 
