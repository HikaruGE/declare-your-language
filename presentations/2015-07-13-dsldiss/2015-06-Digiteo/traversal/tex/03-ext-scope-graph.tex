\section{Extended Scope Graphs}\sectionlabel{extscopegraph}

In this section we recall the formal theory of name resolution of Neron et al.
\cite{NeronTVW-ESOP-2015} consisting of a scope graph model and resolution
calculus, and extend the model with direct imports to model type-dependent name
resolution as introduced in the previous section. 
% For ease of reference, we have
% included the formal rules of the resolution calculus in \Appendix{rescalc}.


\begin{figure}[t]

\begin{minipage}[t]{\hsize}
\begin{boxedminipage}[t]{\hsize}
\textbf{Resolution paths}
\vspace*{-0.4\baselineskip}
$$\begin{array}{rl}
          s & := \dstep{\di{x}{i}} \mid \istep{\ri{x}{i}}{\di{x}{j}} \mid \sistep{S} \mid \pstep\\
          p & := []\ |\ s\ |\ p\cdot p\\
          & \mbox{\rm (inductively generated)} \\[0pt]
          [] \cdot p & = p \cdot [] = p\\
          (p_1 \cdot p_2) &\cdot\ p_3  = p_1 \cdot (p_2 \cdot p_3)
\end{array}$$ 

\textbf{Well-formed paths}

\vspace*{-0.5\baselineskip}

\[
	   \WF(p) \Leftrightarrow p \in \pstep^* \cdot \sistep{\_}^* 
\]
	
\textbf{Specificity ordering on paths}

\medskip
\begin{minipage}{.49\hsize}
  	\infrule{DI}{}{
		\dstep{\_} < \sistep{\_}
	}

\medskip

	\infrule{IP}{}{
		\sistep{\_} < \pstep 
	}

\medskip

	\infrule{DP}{}{
		 \dstep{\_} < \pstep
	}

\end{minipage}
\hspace*{-8mm}
\begin{minipage}{.57\hsize}
    \infrule{Lex1}{
		s_1 < s_2
	}{ 
		s_1\cdot p_1 < s_2 \cdot p_2
	}


\medskip

	\infrule{Lex2}{
		p_1 < p_2
	}{ 
		s \cdot p_1 < s \cdot p_2
	}

\smallskip
  
\end{minipage}


\end{boxedminipage}
\caption{Resolution paths, well-formedness predicate, and specificity
ordering as introduced in \cite{NeronTVW-ESOP-2015}}
\figurelabel{order}
\end{minipage}

\bigskip
\begin{boxedminipage}{\hsize}
\textbf{Edges in scope graph}
\smallskip

	\infrule{P}{
		\P{S_1} = S_2
	}{
		\seeni \vdash \pstep : S_1 \edge S_2
	}

\medskip

    \infrule{I}{
		\ri{y}{i} \in \I{S_1}\setminus\seeni  
		\tab\tab
		\seeni \vdash p : \ri{y}{i} \resolve \di{y}{j}
    	}{
		\seeni \vdash \istep{\ri{y}{i}}{\di{y}{j}} : S_1 \edge \defscope(\di{y}{j}) 
	}

\medskip

        \infrule{D}{
          \si{S_2} \in \IS{S_1}
        }{
          \seeni \vdash \sistep{S_2} : S_1 \edge S_2
        }

\medskip

\textbf{Transitive closure}

%\medskip
       
	\infrule{N}{
		}{
		\seeni \vdash [] : A \medge A
	}

\medskip

	\infrule{T}{
		\seeni \vdash s : A \edge B
		\tab 
		\seeni \vdash p : B \medge C
	}{
		\seeni \vdash s \cdot p : A \medge C
	}

\smallskip

\textbf{Reachable declarations}
\medskip

	\infrule{R}{
		\di{x}{i} \in \D{S'}
		\tab
		\seeni \vdash p : S \medge S'
		\tab 
		\WF(p)
	}{
		\seeni \vdash p \cdot \dstep{\di{x}{i}} : S \reach \di{x}{i}
	}

\medskip  

\textbf{Visible declarations}
\medskip

\infrule{V}{
  \begin{array}{c}
 	\seeni \vdash p : S \reach \di{x}{i}
		\\	
		\forall j,p' (
  		   \seeni \vdash p' : S \reach \di{x}{j} \Rightarrow 
  		   \neg (p' < p)
  		)   
  \end{array}
 	}{
		\seeni \vdash {p} : S \resolve \di{x}{i}
	}

\medskip

\textbf{Reference resolution}

\medskip 

\infrule{X}{
		\rscopeof{\ri{x}{i}}{S}
    \tab
    \{ \ri{x}{i} \} \cup \seeni \vdash p : S \resolve \di{x}{j}
	}{
		\seeni \vdash p : \ri{x}{i} \resolve \di{x}{j}
	}
\end{boxedminipage}
\caption{Resolution calculus from \cite{NeronTVW-ESOP-2015} extended with direct import rule $D$}
\figurelabel{rescalc}



\end{figure}



\subsection{Scope Graphs}

A \emph{scope graph} is a language-independent model for representing the name
binding structure of programs. A scope graph $\G$ is built around three basic
types of nodes derived from the program abstract syntax tree (AST),
\emph{declarations, references, and scopes}:
\begin{itemize}
 \item A \emph{declaration} is an occurrence of an identifier that introduces a
 name. $\di{x}{i}$ denotes the definition of name $x$ at position $i$ in the
 program. 
 We omit the position $i$ when this can be inferred from context.
 $\D{\G}$ denotes the set of references of $\G$.
 \item A \emph{reference} is an occurrence of an identifier referring to a
 declaration. We write $\ri{x}{i}$ for a reference with name $x$ at position
 $i$. 
 Again, the position $i$ may be omitted when it can be inferred from
 context. 
 $\R{\G}$ denotes the set of references of $\G$.
 \item A \emph{scope} is an abstraction over a set of nodes in the AST that
 behave uniformly with respect to name binding. $\S{\G}$ denotes the set of
 scopes of $\G$.
\end{itemize}
Given these sets, a scope graph is defined by the following functions:
\begin{itemize}
 \item Each declaration $d$ in $\DG$ is declared within a scope denoted
 $\scopeof(d)$.
 \item Each declaration $d$ has an optional \emph{associated scope},
 $\defscope(d)$ that is the scope corresponding to the body of the
 declaration. For example, the declarations in a module are elements of its
 associated scope.
 \item Each reference $r$ in $\RG$ is declared within a scope denoted
 $\scopeof(r)$.
 \item Each scope $S$ in $\SG$ has an optional \emph{parent scope} $\P{S}$ that 
 corresponds to its \emph{lexically enclosing scope}. The parent relation has
 to be well-founded, i.e. there is no infinite sequence $S_n$ such that 
 $S_{n+1} = \P{S_n}$.
 \item Each scope $S$ has an associated set of references $\I{S}$, that 
 represents the \emph{imports} in this scope
\end{itemize}
We define by comprehension the set of declarations enclosed in a scope
$S$, as $\D{S} = \{ d \mid \rscopeof{d}{S}\}$.

\subsection{Resolution Calculus}

Given this model, the \emph{resolution calculus} defines the \emph{resolution}
of a reference to a declaration in a scope graph \cite{NeronTVW-ESOP-2015} as
the minimal path from reference to declaration through parent and import edges.
A path $p$ is a list of steps representing the atomic scope transitions in the
graph. A step is either a parent step $\pstep$, an import step
$\istep{\r{y}}{\ds{y}{S}}$ where $\r{y}$ is the imports used and $\ds{y}{S}$ its
corresponding declaration or a declaration step $\dstep{\d{x}}$ leading to a
declaration $\d{x}$.
Given a seen import set $\seeni$, a path $p$ is a valid resolution in the graph
from reference $\ri{x}{i}$ to declaration $\di{x}{i}$ when the following
statement holds:\vspace*{-2mm}
$$\seeni \vdash p : \ri{x}{i} \resolve \di{x}{i}\vspace*{-2mm}$$ 
The calculus in
\Figure{rescalc} defines the resolution relation in terms of edges in the scope
graph, reachable declarations, and visible declarations.

The resolution calculus is parametrized by two predicates on paths, a
\emph{well-formedness predicate} $WF(p)$ and an \emph{ordering relation} $<$
that allows the formalization of different name-binding policies such as
transitive vs non-transitive imports. A typical definition of the
well-formedness predicate is \emph{no-parents-after-imports}, which entails that
a resolution can not proceed to a lexical parent after an import transition.
\Figure{order} presents the definition of paths ($p$) consisting of steps ($s$)
and an example of a path well-formedness predicate and path ordering relation.
This configuration supports arbitrary levels of lexical scope ($\pstep^*$),
transitive imports ($\sistep{\_}^*$), no-parents-after-imports (an $\sistep{\_}$
step cannot be followed by a $\pstep$), prefer local declarations over imported
declarations ($DI$), prefer local declarations over declarations in parents
($DP$), and prefer imported declarations over declarations in parents ($IP$).


\subsection{Direct Imports}

% The scope graph model and resolution calculus provide a large coverage of
% binding constructors in existing programming languages. However, it does not
% support the dependence of resolution of a reference on the \emph{type} of an
% expression. For example, in an object-oriented language, the name \cod{x} in
% \cod{e.x} has to be resolved to the declaration of the field \cod{x} that
% appears in some class definition \cod{A} that is the type of the
% receiver expression \cod{e}.

In order to model type-depdendent name resolution we extend the scope graph with
\emph{direct imports}. A direct import defines a direct link between two scopes
without the use of a reference. In addition to its set of associated imports
(references of the form $\r{x}$), a scope is extended with an associated set of
direct imports $\IS{S}$ consisting of other scopes in the graph. For these
imports we introduce the $(D)$ scope transition rule, which is similar to the
$(I)$ rule of the original calculus, except that this transition does not
require the intermediate resolution of a reference:
\smallskip

\infrule{D}{\si{S_2} \in \IS{S_1}}{\seeni \vdash \sistep{S_2} : S_1 \edge S_2}  

\smallskip\noindent
The complete resolution calculus with this new rule is presented in Figure \ref{fig:rescalc}. 


