\section{Introduction}
\sectionlabel{Introduction}

Name resolution and type resolution are two fundamental concerns in programming
language specification and implementation.
Name resolution means determining the identifier declaration corresponding to
each identifier use in a program.  Type resolution means determining the type of
each identifier and expression in the program, as part of performing type
checking or inference.
These two tasks are essential components of many language processing tools,
including interpreters, compilers and IDEs. Moreover, precise descriptions of
name and type resolution are essential parts of a formal language semantics.
Yet there are as yet no universally accepted formalisms that support both
specification and implementation of these tasks.
This is in notable contrast to the situation with syntax definition, for which
context-free grammars provide a well-established declarative formalism that
underpins a wide variety of useful tools.

In this paper, we show how two existing formalisms, scope graphs and type
constraints, can be extended and combined to fill this gap.
Our formalisms: (i) have a clear and clean underlying theory; (ii) handle a
broad range of common language features; (iii) are declarative, but are
realizable by practical algorithms and tools; (iv) are factored into
language-specific and language-independent parts, to maximize re-use; and (v)
apply to erroneous programs (for which resolution fails or is ambiguous) as well
as to correct ones.   Moreover, although name and type resolution are obviously
related, as far as possible we treat them as separate concerns; this improves
modularity and helps clarify exactly what the relationships between these two
tasks are.

Our starting point is recent work by Neron {\it et
al.}~\cite{NeronTVW-ESOP-2015}, which shows how name resolution for
lexically-scoped languages can be formalized in a way that meets the criteria
above.  The name binding structure of a program is captured in a {\it scope
graph} which records identifier declarations and references and their scoping
relationships, while abstracting away program details.  Its basic building
blocks are {\it scopes}, which are minimal program regions that behave uniformly
with respect to resolution.  Each scope can contain identifier declarations and
references, each tagged with its position in the original AST.
A scope graph is constructed from the program AST using a language-dependent
traversal, but thereafter, it can be processed in a language-independent way.  A
{\it resolution calculus} gives a formal definition of what it means for a
reference to identifier $x$ at position $i$ to resolve to a declaration of $x$
at position $j$, written $\ri{x}{i}\resolve\di{x}{j}$.
A given reference may resolve to one, none, or many declarations.  There is a
sound and complete {\it resolution algorithm} that computes the set of
declarations to which each reference resolves.

Scope graphs do not include explicit type information. However, if the language
associates types with identifier declarations, it is easy to obtain the type of
an identifier reference by first resolving the reference to a declaration and
then looking up the associated type information by position in the
AST.

% \footnote{ There is nothing special about types in this regard: the same
% mechanism can be used to obtain other declaration attributes such as record
% field offsets, visibility attributes, etc.}

One well-known mechanism for type resolution, which meets our formalism criteria
above, is based on extracting {\it constraints} on types and type variables from
the AST and then using {\it unification} to solve the constraints and
instantiate the variables.
This technique goes back at least to Milner's seminal paper on
polymorphism~\cite{Milner78:0}, and has since been extended to cover many
additional language features, notably subtyping. Pottier and Remy
\cite{Pottier-Remy/emlti} give a detailed exposition, and show how an efficient resolution
algorithm can be expressed using rewrite rules.
The constraint approach is most commonly used for type inference, but even for
the simpler problem of type checking, passing to constraints is a useful way to
separate the language-dependent part of the task (generating the constraints)
from the language-independent part (solving the constaints).

This simple two stage approach---name resolution using the scope graph followed
by a separate type resolution stage---will work for many language constructs.
But the full story is more complicated, because sometimes name resolution also
depends on type resolution.
Consider the program fragments in Figure~\ref{fig:records-program}, written in a
language with nominal records and using standard dot notation for record field
access. 
(Subscripts on identifiers represent source code positions and are not part
of the language itself.)
In order to resolve the type of \cod{y$_7$.x$_8$} we must first resolve the
field name \cod{x$_8$} to the appropriate declaration field (\cod{x$_2$} or
\cod{x$_6$}).
But this name resolution depends on the {\it type} of \cod{y$_7$}, so we must
resolve that type first, which again, requires first resolving the \emph{name}
of \cod{y$_7$}.
In general, we may need arbitrarily deep recursion between the two kinds of
resolution. For example, to handle the nested record dereference on the last
line, we must first resolve the name of \cod{y$_9$}, then its type, then the
name and type of \cod{a$_{10}$}, and finally the name and type of
\cod{x$_{11}$}.

To solve this challenge, we reformulate the task of generating a scope graph
from a given program as one of finding a minimal solution to a set of {\it scope
constraints} obtained by an AST traversal.
Scope constraints are analogous to typing constraints, but are resolved using a
different (and simpler) algorithm.
We then introduce a class of {\it scope variables} and modify the resolution
calculus to characterize resolution in potentially {\it incomplete} scope graphs
(i.e., graphs characterized by constraints involving unresolved scope
variables).
We can then interleave (partial) scope graph resolution and type unification
until a complete instantiation of all variables (types, positions, and scopes)
is obtained.  This approach permits us to resolve all the names and types for
the record examples of Figure~\ref{fig:records-program} and for a broad range of
other language constructs.

\paragraph{Contributions}

Our specific contributions are as follows:

\begin{itemize}
\item We show how to complement name resolution based on scope graphs with
  type resolution based on type constraints including
  type-dependent name resolution (\Section{by-example}).
\item We extend the name resolution calculus and algorithm of 
\cite{NeronTVW-ESOP-2015} to handle incomplete scope graphs 
(\Section{extscopegraph}, \Section{resolution-algorithm}).
\item We define a constraint language that can express both typing and name
binding constraints, parameterized by an underlying notion of type
compatibility, and define satisfiability for problems in this language
(\Section{constraintlang}).
\item We describe the details of constraint generation for a model
language that illustrates many of the interesting resolution issues
associated with modules, classes, and records (\Section{collection}). 
\item We describe an algorithm for solving problems in our constraint
language instantiated to use nominal subtyping, and show that it is 
sound with respect to the satisfiability
specification (\Section{resolution-algorithm}).
\end{itemize}

% \item (and complete?????? \PN{Unlikely for the generic
% constraint language since variables can be everywhere, maybe if we restrict variable to where they
% actually appear  in our collection}). 

\noindent Our constraint generator and solver have been implemented and will be
submitted as artifacts to accompany the paper. The implementation provides name
and type resolution in the IDE generated with the Spoofax Language
Workbench~\cite{KatsV10} for the LMR model language used in this paper and has
been used to generate the scope graphs and type constraints for the examples in
this paper automatically.

\begin{figure}[t]
\begin{lstlisting}[language=LMR,basicstyle=\lstfigurestyle,breaklines=true]
record A$_1${ x$_2$: Int }
record B$_3${ a$_4$: A$_5$  x$_6$:Bool }
...
y$_7$.x$_8$  // what is the type of y$_7$ ? 
y$_9$.a$_{10}$.x$_{11}$ // what are the types of y$_9$, y$_9$.a$_{10}$ ?
\end{lstlisting}
\caption{Program with records}
\label{fig:records-program} 
\end{figure}


\endinput

As an example, consider the program in Figure~\ref{fig:simple-prog}
in an expression-oriented language with standard lexical scoping rules.
\begin{figure}
\begin{lstlisting}[language=LMR,basicstyle=\lstfigurestyle,breaklines=true]
fun y$_1$:Int { 
  fun y$_2$:Bool { 
    let x$_3$:Bool = if y$_4$ then True else y$_5$ 
    in x$_6$
  }
}
\end{lstlisting}
\label{fig:simple-prog}
\caption{A simple program}
\end{figure}
Subscripts on identifiers
represent AST positions. From the scope graph for this program,
the resolution algorithm will generate the facts
\[
\ri{\cod{y}}{4}  \resolve  \di{\cod{y}}{2},
\ri{\cod{y}}{5}  \resolve  \di{\cod{y}}{2},
\ri{\cod{x}}{6}  \resolve  \di{\cod{x}}{3}
\]
Note that resolution takes account of the shadowing of the binding
at \cod{y}$_1$ by the binding at \cod{y}$_2$; this is the whole point
of the resolution calculus.

Scope graphs do not 
include explicit type information. However, if the language associates
types with identifier declarations, it is easy to obtain the type of
an identifier reference by first resolving the reference to a declaration
and then looking up the associated type information by position in the AST.\footnote{
There is nothing special about types in this regard: the same
mechanism can be used to obtain other declaration attributes such as 
record field offsets, visibility attributes, etc.} 
For example, the program of Figure~\ref{fig:simple-prog} carries this
associated type information:
\[
\typeat{1}  =  \mbox{\tt Int},
\typeat{2}  =  \mbox{\tt Bool},
\typeat{3}  =  \mbox{\tt Bool}
\]
This lookup facility 
can be made available as a service to a separate type resolution mechanism.

One well-known mechanism for type resolution, which meets our formalism criteria
above, is based on extracting {\it constraints} on types and type variables from
the AST and then using {\it unification} to solve the constraints and
instantiate the variables.
This technique goes back at least to Milner's seminal paper on
polymorphism~\cite{Milner78:0}, and has since been extended to cover many
additional language features, notably subtyping; Pottier and Remy~(Chapter 10 of
\cite{Pierce05advancedtopics}) give a detailed exposition, and show how an
efficient resolution algorithm can be expressed using rewrite rules.
The constraint approach is most commonly used for type inference, but even for
the simpler problem of type checking, passing to constraints is a useful way to
separate the language-dependent part of the task (generating the constraints)
from the language-independent part (solving the constaints).

Figure~\ref{fig:simple-constraints} illustrates how a constraint-based approach to
type checking might work on top of name resolution for a small fragment
of a simple monomorphically-typed language. At the top of the figure,
we show the set of constraints, obtained by walking over the program AST; 
we defer discussion of the details of constraint generation to a later section.
\begin{figure}[tb]
\begin{boxedminipage}{\hsize}
\center{
Constraints (where overall program type is $\tau_0$)}
\[
\begin{array}{rcl}
\tau_0 & = & \mbox{\tt Int} \rightarrow \tau_1  \\
\tau_1 & = & \mbox{\tt Bool} \rightarrow \tau_2  \\
\ri{y}{4} & \resolve & \di{y}{\iota_1} \\
\typeat{\iota_1} & = & \mbox{\tt Bool}\\
\ri{y}{5} & \resolve & \di{y}{\iota_2} \\
\typeat{\iota_2} & = & \mbox{\tt Bool} \\
\ri{x}{6} & \resolve & \di{x}{\iota_3} \\
\typeat{\iota_3} & = & \tau_2\\
\end{array}
\]
\end{boxedminipage}
\\
\begin{boxedminipage}{\hsize}
\center{Solution to constraints}
\\
$\iota_1 = 2$, $\iota_2 = 2$, $\iota_3 = 3$,
$\tau_2 = \mbox{\tt Bool}$,\\
$\tau_1 = \mbox{\tt Bool} \rightarrow \mbox{\tt Bool}$,
$\tau_0 = \mbox{\tt Int} \rightarrow \mbox{\tt Bool} \rightarrow \mbox{\tt Bool}$
\end{boxedminipage}
\caption{Constraints for Simple Program}
\label{fig:simple-constraints}
\end{figure}
There are two forms of constraints: conventional type equalities,
and name resolution constraints.  The latter replace the implicit
name resolution via manipulation of a typing context that would 
be used in a more conventional presentation.
Note that constraints can
mention {\em position variables} $\iota$ as well as ordinary type variables $\tau$. At the bottom of the figure, we show a solution to the constraint
set, written as a substiution on type and position variables. The solution
is obtained by plugging in the facts about $\resolve$ and $\typeat{}$ derived
from the scope graph, and then performing unification; again, we defer details
to a later section.

%% Source expression:
%% \\
%% {\tt $\lambda$ y$_1$:Int. $\lambda$ y$_2$:Bool. let x$_3$:Bool = if y$_4$ then True else y$_5$ in x$_6$}
%% \\
%% \\
%% Facts from scope graph:
%% \\
%% $
%% \begin{array}{rcl}
%% \ri{y}{4} & \resolve & \di{y}{2} \\
%% \ri{y}{5} & \resolve & \di{y}{2} \\
%% \ri{x}{6} & \resolve & \di{x}{3} \\
%% \end{array}
%% $
%% \\
%% Facts from AST:
%% \\
%% $
%% \begin{array}{rcl}
%% \typeat{1} & = & \mbox{\tt Int}\\
%% \typeat{2} & = & \mbox{\tt Bool}\\
%% \typeat{3} & = & \mbox{\tt Bool}
%% \end{array}
%% $
%% \\

%% The language of constraints:

%% t_1 = t_2

%% where

%% t = Int | Bool | t -> t | TypeAt(p) | TypeOf(c) | \tau

%% and positions p = i | \iota

%% \semop{if e1 then e2 else e3}^t  =  \semop{e1}^bool /\  [[e2]]^t /\ [[e3]]^t  

%% [[x_i]]^t = x_i |-->  x_j /\ TypeAt(j) = t  (fresh $j$)

%% [[c]]^t = TypeOf(c) = t

%% [[let x_i:u = e1 in e2]]^t = [[e1]^u  /\ [[e2]]^t   

%% [[\x_i:u.e]]^t = t = u -> \tau /\ [[e]]^\tau              (fresh \tau)


This simple two stage approach---name resolution using the
scope graph followed by separate type resolution---will work for
many language constructs. But the full story is more complicated,
because sometimes name resolution also depends on type resolution.
Consider the program fragments in Figure~\ref{fig:records-program}, written
in a language having nominal records and using standard dot notation 
for record field access. 
\begin{figure}
\begin{lstlisting}[language=LMR,basicstyle=\lstfigurestyle,breaklines=true]
record A$_1$ {x$_2$:Int}
record B$_3$ {a$_4$:A$_5$, x$_6$:Bool}
...
y$_7$.x$_8$  // what is the type of y$_7$ ? 
y$_9$.a$_{10}$.x$_{11}$ // what are the types of y$_9$, y$_9$.a$_{10}$ ?
\end{lstlisting}
\caption{Program with records}
\label{fig:records-program} 
\end{figure}
In a scope graph setting, we can create one scope that contains
the field declaration of record type \cod{A$_1$}, and another that
contains those of \cod{B$_3$}, and associate each scope with the
corresponding type name.
Now, in order to resolve the type of \cod{y$_7$.x$_8$} we must first 
resolve the field name \cod{x$_8$} to the appropriate
declaration field (\cod{x$_2$} or \cod{x$_6$}). 
But this name resolution depends on the {\it type} of \cod{y$_7$},
so we must resolve that type first.  
In general, we may need arbitrarily deep recursion between
the two kinds of resolution. For example, to handle the nested 
record dereference on the last line, we must first resolve the name
of \cod{y$_9$}, then its type, then the name and type of \cod{a$_{10}$}, 
and finally the name and type of \cod{x$_{11}$}. 

To solve this challenge, we 
reformulate the task of generating a scope graph from a given program
as one of finding a minimal solution to a set of {\it scope constraints}
obtained by an AST traversal.
Scope constraints are analogous to typing constraints, 
but are resolved using a different (and simpler) algorithm.
We then introduce a class of {\it scope variables} and modify
the resolution calculus to characterize resolution in potentially {\it incomplete} 
scope graphs (i.e., graphs characterized by constraints involving unresolved 
scope variables).
We can then interleave (partial) scope graph resolution and type unification 
until a complete instantiation of all variables (types, positions, and
scopes) is obtained.  This approach permits us to resolve all the names
and types for the record examples of Figure~\ref{fig:records-program} 
and for a broad range of other language constructs. 

Our specific contributions are as follows:

\begin{itemize}
\item We define a constraint language that can express both typing 
and name binding constraints, parameterized by an underlying notion
of type compatibility, and define satisfiability for problems
in this language. 
\item We show how to extend the name resolution calculus and algorithm of 
~\cite{NeronTVW-ESOP-2015} to handle incomplete scope graphs.
\item We describe details of constraint generation for a simple toy
language that illustrates many of the interesting resolution issues
associated with modules, classes, and records. 
\item We describe an algorithm for solving problems in our constraint
language instantiated to use nominal subtyping, and show that it is 
sound (and complete?????? \PN{Unlikely for the generic constraint language since variables can be everywhere, maybe if we restrict variable to where they actually appear in our collection}).
\end{itemize}
Our constraint generator and solver have been implemented and will
be submitted as artifacts to accompany the paper.

[Roadmap may follow or be incorporated in above.]

%% \paragraph{Contributions:}
%% \begin{itemize}
%%  \item extension of scope graphs with types
%%  \item general theory of name and type resolution in this extended model
%%  \item resolution algorithm for several instances
%%  \item a prototype implementation of resolution alg(s)
%% \end{itemize}


%% In this paper, we extend the scope graph formalism to handle type dependent resolution.

%% the formalism is parameterized with a theory of equality on types (HM(X)-ish)

%% we present a general theory of combined name and type resolution

%% we derive an algorithm for name and type resolution for several instances, including monomorphic types, nominal subtyping, Hindley-Milner polymorphism, …

%% abstract syntax trees are translated to constraints that represent (partial) facts about name and types of a program

%% Introduced name binding as separate concern in \cite{NeronTVW-ESOP-2015}
%% Name binding can use type system e.g. a.x where a is a record or an object
%% Type systems use name binding usually through environments, but environment can become complicated structures (with OO languages)

%% Focus on nominal type system ?

%% Goal: \\
%% Express typing rules using the resolution relation from \cite{NeronTVW-ESOP-2015}\\
%% Extend coverage of scope graphs with interactions between type inference and name resolution (A x = New A; A.x ; Pascal ``with'')

%% \paragraph{Contributions:}
%% \begin{itemize}
%%  \item Constraint language to express both typing and name binding constraints
%%  \item A name resolution specification and algorithm for incomplete scope graphs
%%  \item Constraint generation for a language LMR with modules and records
%%  \item Constraint solving algorithm 
%% \end{itemize}


%% \paragraph{Contributions:}
%% \begin{itemize}
%%  \item extension of scope graphs with types
%%  \item general theory of name and type resolution in this extended model
%%  \item resolution algorithm for several instances
%%  \item a prototype implementation of resolution alg(s)
%% \end{itemize}

%% \paragraph{Motivation}

%% many kinds of interaction between name resolution and type resolution

%% we would like to propose a general framework for modeling such interactions

%% \paragraph{eventual goal:} 
%% a single framework that can be used to declaratively describe name and type resolution, as basis for reasoning and implementation.

%% \paragraph{dimensions of generality:}

%% language independent modeling of interaction phenomena

%% strength of the type system: monomorphic types, subtyping, parametric polymorphism, non-parametric overloading (type classes), structural types

%% \paragraph{features we may / should consider to support (in the limit):}
%% \begin{itemize}
%%  \item qualified names
%%  \item type-based name resolution (e.g. field access in Java): resolution scopes based on types
%%  \item with in Pascal: imports based on types
%%  \item overloading resolution in Java : filtering of appropriate names based on types
%%  \item nominal subtyping: based on declarations of classes and their supertypes
%%  \item Hindley/Milner parametric polymorphism
%%  \item generics in Java : HM + subtyping + bounded parameters
%%  \item functors / signatures in ML
%%  \item structural types; resolution into types that are computed during analysis
%%  \item record types     
%%  \item structural subtyping
%%  \item interfaces in Java: check that implementation is complete
%%  \item traits
%%  \item type classes
%% \end{itemize}

%% \paragraph{Scope Graphs with Types}

%% formalization of the representation of scope graphs with types using constraints

%% specification of the semantics by means of deduction rules (?)

%% type dependent resolution


