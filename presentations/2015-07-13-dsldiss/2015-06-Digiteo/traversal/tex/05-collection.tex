\section{Constraint Collection}
\sectionlabel{collection}

\newcommand{\kind}[1]{\footnotesize{\mbox{\it{#1}}}}

\newcommand{\sema}[3]{
\semop{#2}^{\mathit{#1}}_{#3}
}

\newcommand{\fresh}[1]{\mbox{\it(fresh ${#1}$)}} 
\newcommand{\new}[1]{\mbox{\it(new ${#1}$)}} 

\newcommand{\cand}{\ \wedge\ }  % was comma

\newcommand{\inttysub}{\footnotesize \intty}
\newcommand{\arrtysub}[2]{\footnotesize {\arrty{#1}{#2}}}
\newcommand{\rectysub}[1]{\footnotesize {\recty{#1}}}

In this section, we show how to collect constraints for name resolution and
typechecking from programs in the LMR language, whose concrete syntax was
given in \Figure{lmr:grammar}.
The full collection algorithm is shown in Figures~\ref{fig:consgen} and \ref{fig:consgen2}.  
\begin{figure*}[t]
%\begin{boxedminipage}[t]{\hsize}
\small
$\begin{array}{rclr}
\sema{\w{prog}}{\w{ds}}{} & 
\sdefine &
\AParent{S}{\perp} \cand \sema{\ks{\w{decl}}}{\w{ds}}{S}
& \new{S} 
\\[1.5ex]
\sema{\w{decl}}{\kw{module}\ X_i\ \sep{\{} \w{ds} \sep{\}}}{s} & \sdefine &  %
\AD{\dsi{X}{i}{S'}}{s}
\cand  \AParent{S'}{s}
\cand \sema{\ks{\w{decl}}}{\w{ds}}{S'}
& \new{S'} 
\\[1.5ex]
\sema{\w{decl}}{\kw{import}\ \w{Xs}\sep{.} X_i}{s} & 
\sdefine &
\multicolumn{2}{l}{
\ARImport{\ri{X}{i}}{s}  
\cand \sema{\w{qid}}{\w{Xs}\sep{.} X_i}{s} 
}
\\[1.5ex]
\sema{\w{decl}}{\kw{def}\ b}{s} & 
\sdefine &
\sema{\w{bind}}{b}{s,s}
\\[1.5ex]
\sema{\w{bind}}{x_i\ \sep{=}\ e}{s_r,s_d} & 
\sdefine &
\AD{\di{x}{i}}{s_d}
\cand  \sema{\w{exp}}{e}{s_r}  
\\[1.5ex]
\sema{\w{qid}}{x_i}{s} & 
\sdefine &
\AR{\ri{x}{i}}{s} 
%\\[1.5ex]
%\sema{\w{qid}}{\w{Xs} \sep{.} X_j \sep{.} x_i}{s} & 
%\sdefine &
%\AR{\ri{x}{i}}{S'} 
%\cand \AParent{S'}{\undef} 
%\cand \ARImport{\ri{X}{j}}{S'}
%\cand \sema{\w{qid}}{\w{Xs} \sep{.} X_j}{s} 
%& \new{S'} 
\\[1.5ex]
\sema{\w{exp}}{\kw{fun}\ \sep{(} x_i \sep{:} t \sep{)} \sep{\{} e \sep{\}}}{s} & 
\sdefine &
\AParent{S'}{s}
\cand \AD{\di{x}{i}}{S'} 
\cand \sema{\w{exp}}{e}{S'}  
& \new{S'}
\\[1.5ex]
\sema{\w{exp}}{\kw{letrec}\ \w{bs}\ \kw{in}\ e}{s} & 
\sdefine &
\AParent{S'}{s} 
\cand \sema{\ks{\w{bind}}}{\w{bs}}{S',S'} 
\cand \sema{\w{exp}}{e}{S'} 
& \new{S'} 
\\[1.5ex]
\sema{\w{exp}}{\kw{letpar}\ \w{bs}\ \kw{in}\ e}{s} & 
\sdefine &
\AParent{S'}{s}
\cand  \sema{\ks{\w{bind}}}{\w{bs}}{S,S'} 
\cand \sema{\w{exp}}{e}{S'} 
& \new{S'}
\\[1.5ex]
\sema{\w{exp}}{\w{Xs}\sep{.} x_i}{s} & 
\sdefine &
\sema{\w{qid}}{\w{Xs}\sep{.} x_i}{s} 
\\[1.5ex]
\sema{\w{exp}}{e_1\ e_2}{s} & 
\sdefine &
\sema{\w{exp}}{e_1}{s}
\cand \sema{\w{exp}}{e_2}{s} 
\end{array}$
%\end{boxedminipage}
\end{figure*}

\begin{figure*}[!t]
\begin{boxedminipage}[t]{\hsize}
\small
\centering
$\begin{array}{rclr}
\sema{\w{exp}}{n}{s,t} & 
\sdefine &
\CTeq{t}{\intty} 
\\[1.5ex]
\sema{\w{exp}}{\kw{true}}{s,t} & 
\sdefine &
\CTeq{t}{\boolty}  
\hspace{3em}
\begin{array}{rclr}
\sema{\w{exp}}{\kw{false}}{s,t} &
\sdefine &
\CTeq{t}{\boolty} 
\end{array}
\\[1.5ex]
\sema{\w{exp}}{e_1\ \oplus\ e_2}{s,t} & 
\sdefine &
\CType{\oplus}{\arrty{\tau_1}{\arrty{\tau_2}{\tau_3}}}
\cand \CTeq{t}{\tau_3} 
\cand \sema{\w{exp}}{e_1}{s,\tau_1} 
\cand \sema{\w{exp}}{e_2}{s,\tau_2}
& \fresh{\tau_1, \tau_2, \tau_3}
\\[1.5ex]
\sema{\w{exp}}{\kw{if}\ e_1\ \kw{then}\ e_2\ \kw{else}\ e_3}{s,t} &
\sdefine &
\CLub{t}{\tau_2}{\tau_3}
\cand \sema{\w{exp}}{e_1}{s,\boolty}
\cand
\sema{\w{exp}}{e_2}{s,\tau_2}
\cand \sema{\w{exp}}{e_3}{s,\tau_3}
& \fresh{\tau_2,\tau_3}
\\[1.5ex]
\sema{\w{exp}}{\w{Xs}\sep{.} x_i}{s,t} & 
\sdefine &
\CResolve{\ri{x}{i}}{\delta} 
\cand \CType{\delta}{t} 
\cand \sema{\w{qid}}{\w{Xs}\sep{.} x_i}{s} 
& \fresh{\delta}
\\[1.5ex]
\sema{\w{exp}}{e_1\ e_2}{s,t} & 
\sdefine &
\CSubtype{\tau_2}{\tau_1} 
\cand \sema{\w{exp}}{e_1}{s,\arrtysub{\tau_1}{t}}
\cand \sema{\w{exp}}{e_2}{s,\tau_2} 
& \fresh{\tau_1,\tau_2} 
\\[1.5ex]
% % sequential let omitted
\sema{\w{exp}}{e \sep{.} x_i}{s,t} & 
\sdefine &
\AParent{S'}{\undef} 
\cand \ASImport{\varsigma}{S'}
\cand \AScopeof{\ri{x}{i}}{S'}
\cand \CAssoc{\delta_1}{\varsigma}
\cand \CResolve{\ri{x}{i}}{\delta_2}
& \new{S'}\fresh{\delta_1,\delta_2,\varsigma}
\\
& \cand & \CType{\delta_2}{t}
\cand \sema{\w{exp}}{e}{s,\recty{\delta_1}} 
\\[1.5ex]
\sema{\w{exp}}{\kw{with}\ e_1\ \kw{do}\ e_2}{s,t} & 
\sdefine &
\AParent{S'}{s}
\cand \ASImport{\varsigma}{S'}
\cand \CAssoc{\delta}{\varsigma} 
\cand \sema{\w{exp}}{e_1}{s,\recty{\delta}} 
\cand \sema{\w{exp}}{e_2}{S',t}
& \new{S'}\fresh{\delta,\varsigma}
\\[1.5ex]
\sema{\w{exp}}{\kw{new}\ \w{Xs}\sep{.} X_i\ \sep{\{} \w{bs} \sep{\}}}{s,t} & \sdefine &
\AParent{S'}{s} 
\cand \ARImport{\ri{X}{i}}{S'} 
\cand \CResolve{\ri{X}{i}}{\delta} 
\cand \CTeq{t}{\recty{\delta}} 
   & \new{S'}\fresh{\delta}\\
& \cand & \sema{\w{qid}}{\w{Xs} \sep{.} X_i}{s} 
\cand \sema{\ks{\w{fbind}}}{bs}{s,s'}
\\[1.5ex]
\sema{\w{fbind}}{x_i\ \sep{=}\ e}{s_r,s_d} & 
\sdefine &
\AScopeof{\ri{x}{i}}{s_r}
\cand \CResolve{\ri{x}{i}}{\delta} 
\cand \CType{\delta}{\tau_1}
\cand \CSubtype{\tau_2}{\tau_1} 
\cand \sema{\w{exp}}{e}{s_r,\tau_2}
& \fresh{\delta, \tau_1, \tau_2}
\end{array}$
\end{boxedminipage}
  \caption{Constraint generation for LMR.}
  \label{fig:consgen2}
\end{figure*}

Collection is performed by a single traversal over 
the program that collects scope and subtyping assumptions, name resolution 
constraints, and typing constraints all in one pass. (The color codings 
should help in distinguishing these different kinds of constraints.)

To simplify and compress the presentation, we describe the algorithm as operating over LMR's
concrete syntax. (Our actual implementation operates over the abstract
syntax of LMR, and is written in \DynSem{}, a declarative domain-specific 
language for expressing semantics; although readable, it is relatively verbose.)
The algorithm is defined by a family of functions indexed by
syntactic category ({\it decl}, {\it exp}, etc.). Each function takes a syntactic item and
possibly one or more auxiliary parameters, and (usually) 
returns a constraint, possibly involving one or more fresh variables or new
scope identifiers.  Functions are defined by a set of rules, one for each possible syntactic form
in the category. For example, $\sema{decl}{-}{s}$ has four rules (for \cod{module}, \cod{import}, \cod{def}
and \cod{record} declarations, respectively), and is parameterized by
the scope $s$ into which declared identifiers are to be installed; it returns the 
conjunction of constraints that enforces correct name and type resolution for the declaration, 
some of which are derived by invoking generation functions on syntactic sub-components.

To further streamline the presentation, we use the notation 
$\sema{c^*}{-}{}$ on sequences of items of syntactic category $c$ 
to mean the result of applying $\sema{c}{-}{}$ to each item and returning
the conjunction of the resulting constraints, or $\CTrue$ for the empty sequence.
Similarly, $\sema{c?}{-}{}$ works on a optional $c$ item; it applies 
$\sema{c}{-}{}$ to the item if it is present and returns $\CTrue$ otherwise.
Throughout, we use metavariable $x_i$ for a (lower case) term variable at position $i$
and $X_i$ for an (upper case)  module or record name at position $i$, with
one exception: for compactness, we give just one rule for both $qid$ and $Qid$, in which  
$x_i$ can be either kind of identifier.
We write $Xs$ for a dot-separated sequence of
module or record names, which can be empty (in which case, by convention, $Xs.x$ 
doesn't have a leading dot). 

Let us trace how the constraint generator works on some of the different syntactic forms of LMR.
A complete program is a sequence of mutually-recursive top-level declarations, 
so $\sema{prog}{-}{}$ creates a new root scope $S$
in which they are to be installed,  generates an assumption constraint
that $S$ is parentless, and then conjoins the constraints for each declaration, 
passing $S$ as a parameter to the declaration generator for {\it decl}. 
$\sema{decl}{-}{s}$ generates an assumption that installs any declared identifier into scope $s$;
the rest of its behavior depends on the kind of declaration:

A \cod{module} builds a new lexical child scope for its declarations, so the generator
function creates a new scope $S'$ and generates assumptions that $s$ is the parent of $S'$ and
that the module name declaration is associated with $S'$. It then conjoins the constraints obtained
by recursively invoking $\sema{decl}{-}{S'}$ on its member declarations.

An \cod{import} doesn't declare an identifier, but instead generates constraints forcing
the imported module name into the import set and reference set for $s$. The $\sema{qid}{-}{s}$ 
invocation generates additional constraints needed to describe references to potentially qualified names.
(We omit discussion of the details, which are slightly complex.)

A \cod{def} invokes an auxiliary generating function $\sema{bind}{-}{s_r,s_d}$ to process the
definition; $s_d$ is the scope into which the defined identifier's declaration should be installed, and $s_r$
is the scope into which any identifier references in the definining expression should go. (In this invocation,
the two scope parameters are the same, but the {\it bind} generator is invoked at other places, e.g.
for the \cod{letpar} expression, where they are not.)  For bindings without explicit type annotation,
a constraint is generated giving the defined identifier a fresh type $\tau$, and function $\sema{exp}{-}{s,t}$
is invoked to generate constraints for the defining expression with the expected type $t=\tau$.
(We discuss generation of expression constraints below.)
For bindings with an explicit concrete type annotation, we also generate a constraint that $\tau$ be a subtype
of the declared type, after it is translated into an internal type constructor
using auxiliary function $\sema{ty}{-}{s}$. This function, unlike all the others, returns a pair of things:
the internalized type constructor, and any constraints generated by references to record names (which are
installed into the scope given by parameter $s$).

A \cod{record} is handled similarly to a \cod{module}, but the details are more complicated. If the
record has a super-type (non-empty \cod{extends} clause), an auxiliary function $\sema{sup}{-}{s_r,s_d,t}$ 
is invoked to generate constraints to describe the inheritance.
Parameter $s_r=s$ is the scope in which the super-type name is to be resolved,
$s_d=S'$ is the scope into which the super-type's scope is to be imported, and $t=\recty{\di{X}{i}}$ 
is the new record type. The {\it sup} generator builds a subtyping assumption (the only
source of such assumptions in LMR) that relates the new record type to the result of resolving the
super-type name, via a fresh declaration variable $\delta$. The fields of the new record are
declared and given type constraints by another auxiliary function.


Constraint generation for expressions is largely straightforward. The {\it expr} 
generator is parameterized by the scope $s$ into which references should be installed
and the expected type $t$ of the expression (often a type variable). Expressions 
introducing local bindings re-use the {\it bind} generator. (There is
function for sequential \cod{let}, which is desugared into nested \cod{letpar}
expressions before constraint generation is performed.)
Note that generated constraints always force an expression to have a precise type, which is
designed to be minimal in the subtyping hierarchy. Subtyping is allowed only at
function applications, at bindings to explicitly annotated identifiers, and in
the conditional expressions, for which a least-upper-bound constraint is generated.
Scope variables $\varsigma$ are introduced only for field dereference and \cod{with}
expressions. 

%% \begin{figure*}

%% \centering
%% \renewcommand{\arraystretch}{1.4}
%% $
%% \begin{array}{rclr}

%% \sema{\kind{prog}}{\semcode{$ds$}}{} & \sdefine &
%%   \AParent{S}{\perp} \cand \sema{\kind{decl}^*}{\semcode{$ds$}}{S} & 
%%   \new{S} \\[0.3cm]
%% %% \sema{decls}{\semcode{$d\ ds$}}{s} & \sdefine &
%% %%   \sema{decl}{\semcode{$d$}}{s} \cand \sema{decls}{\semcode{$ds$}}{s} \\
%% %% \sema{decls}{}{s} & \sdefine &
%% %%   \CTrue \\
%% %\\
%% \sema{\kind{decl}}{\semcode{module\ $X_i$\ \{ $ds$ \}}}{s} & \sdefine &  %
%%   \AScopeof{\di{X}{i}}{s} \cand
%%   \AParent{S'}{s} \cand
%%   \AAssoc{\di{X}{i}}{S'} \cand
%%   \sema{\kind{decl}^*}{\semcode{$ds$}}{S'}  &
%%   \new{S'} \\
%% \sema{\kind{decl}}{\semcode{import\ $Xs$.$X_i$}}{s} & \sdefine &
%%   \sema{\kind{qid}}{\semcode{$Xs$.$X_i$}}{s} \cand
%%   \ARImport{\ri{X}{i}}{s}  \\
%% \sema{\kind{decl}}{\semcode{def $\ b$}}{s} & \sdefine &
%%    \sema{\kind{defbind}}{\semcode{$b$}}{s,s}  \\
%% \sema{\kind{decl}}{\semcode{record\ $X_i\ sup\ $ \{$\mbox{\it fs}$\}}}{s} & \sdefine &
%%   \AScopeof{\di{X}{i}}{s} \cand
%%   \AParent{S'}{s} \cand
%%   \AAssoc{\di{X}{i}}{S'} \cand \\
%%   & & % \CType{\di{X}{i}}{Rec(\di{X}{i})} \cand 
%%   \sema{\kind{super?}}{\semcode{$sup$}}{s,S',\rectysub{\di{X}{i}}} \cand 
%%   \sema{\kind{flddecl}^*}{\semcode{$\mbox{\it fs}$}}{s,S'} & 
%%   \new{S'} \\[0.3cm]
%% %\\
%% \sema{\kind{super}}{\semcode{extends\ $Xs$.$X_i$}}{s_r,s_d,t} & \sdefine &
%%   \sema{\kind{qid}}{\semcode{$Xs$.$X_i$}}{s_r} \cand
%%   \CResolve{\ri{X}{i}}{\delta} \cand
%%   \ARImport{\ri{X}{i}}{s_d} \cand
%%   \ASubtype{t}{\recty{\delta}} 
%% & \fresh{\delta} \\[0.3cm]
%% %% \sema{super}{}{s,s',t} & \sdefine &
%% %%  \CTrue \\
%% % APT: differs in DS
%% %\\
%% %% \sema{flddecls}{\semcode{$x_i$:$t$,$fs$}}{s_r,s_d} & \sdefine &
%% %%   C \cand 
%% %%   \sema{flddecls}{\semcode{$fs$}}{s_r,s_d} \\
%% %%   && \mbox{\it where } \sema{argdecl}{\semcode{$x_i$:$t$}}{s_r,s_d} = (\_,C) \\
%% %% \sema{flddecls}{}{s_r,s_d} &\sdefine &
%% %%   \CTrue \\
%% %% \\
%% \sema{\kind{flddecl}}{\semcode{$x_i$:$t$}}{s_r,s_d} & \sdefine &
%%   \AScopeof{\di{x}{i}}{s_d} \cand \CType{\di{x}{i}}{t'} \cand C \ \ \ \ \ \mbox{\it where } \sema{\kind{ty}}{t}{s_r} = (t',C) \\[0.3cm]
%% %\\
%% %% \sema{qid}{\semcode{$x_i$}}{s} & \sdefine &
%% %%  \AScopeof{\ri{x}{i}}{s} \\
%% %%\sema{qid}{\semcode{$xs$.$x_i$}}{s} & \sdefine &
%% %%  \sema{qid}{\semcode{$xs$}}{s} \cand
%% %%  \AParent{S'}{\undef} \cand
%% %%  \ARImport{\r{last(\semcode{$xs$})}}{S'} \cand
%% %%  \AScopeof{\ri{x}{i}}{S'} & \new{S'} \\
%% %%% \mbox{\rm (or, arguably better:)} \\
%% %% \sema{\kind{qid}}{\semcode{$x_i$}}{s} & \sdefine &
%% %%   \AScopeof{\ri{x}{i}}{s} \\
%% %% \sema{\kind{qid}}{\semcode{$X_i$.$xs$}}{s} & \sdefine &
%% %%   \AScopeof{\ri{X}{i}}{s} \cand
%% %%   \AParent{S'}{\undef} \cand
%% %%   \sema{\kind{qid}}{\semcode{$xs$}}{S'} \cand
%% %%   \ARImport{\ri{X}{i}}{S'}
%% %%   & \new{S'} \\[0.3cm]
%% % or, the latest:
%% \sema{\kind{qid}}{\semcode{$x_i$}}{s} & \sdefine &
%%   \AScopeof{\ri{x}{i}}{s} \\
%% \sema{\kind{qid}}{\semcode{$Xs$.$X_j$.$x_i$}}{s} & \sdefine &
%%   \AParent{S'}{\undef} \cand
%%   \AScopeof{\ri{x}{i}}{S'} \cand
%%   \ARImport{\ri{X}{j}}{S'} \cand
%%   \sema{\kind{qid}}{\semcode{$Xs$.$X_j$}}{s} 
%%   & \new{S'} \\[0.3cm]
%% %\\
%% %%\sema{defbinds}{\semcode{$b\ bs$}}{s_r,s_d} & \sdefine &
%% %%   \sema{defbind}{\semcode{$b$}}{s_r,s_d} \cand \sema{defbinds}{\semcode{$bs$}}{s_r,s_d} \\
%% %%\sema{defbinds}{}{s_r,s_d} & \sdefine & \CTrue \\
%% %% \\
%% \sema{\kind{defbind}}{\semcode{$x_i\ $=$\ e$}}{s_r,s_d} & \sdefine &
%%    \AScopeof{\di{x}{i}}{s_d} \cand
%%    \CType{\di{x}{i}}{\tau} \cand
%%    \sema{\kind{exp}}{\semcode{$e$}}{s_r,\tau}   & \fresh{\tau} \\
%% \sema{\kind{defbind}}{\semcode{$x_i$:$t\ $=$\ e$}}{s_r,s_d} & \sdefine &
%%    \AScopeof{\di{x}{i}}{s_d}\cand
%%    \CType{\di{x}{i}}{t'} \cand
%%    \sema{\kind{exp}}{\semcode{$e$}}{s_r,\tau} \cand
%%    \CSubtype{\tau}{t'} \cand
%%    C \\
%%     & & \mbox{\it where } \sema{\kind{ty}}{t}{s_r} = (t',C) & \fresh{\tau} \\[0.3cm]
%% %\\
%% \sema{\kind{exp}}{\semcode{$n$}}{s,t} & \sdefine &
%%   \CTeq{t}{\intty} \\
%% \sema{\kind{exp}}{\semcode{$e_1\ \oplus\ e_2$}}{s,t} & \sdefine &
%%   \sema{\kind{exp}}{\semcode{$e_1$}}{s,\inttysub} \cand \sema{\kind{exp}}{\semcode{$e_2$}}{s,\inttysub} \cand \CTeq{t}{\intty} \\
%% \sema{\kind{exp}}{\semcode{ifz $\ e_1\ $ then $\ e_2\ $ else $\ e_3$}}{s,t} & \sdefine &
%%   \sema{\kind{exp}}{\semcode{$e_1$}}{s,\inttysub} \cand \sema{\kind{exp}}{\semcode{$e_2$}}{s,\tau_2}  \cand \sema{\kind{exp}}{\semcode{$e_3$}}{s,\tau_3}  \cand
%%   \CLub{t}{\tau_2}{\tau_3} & \fresh{\tau_2,\tau_3} \\ 
%% \sema{\kind{exp}}{\semcode{$Xs$.$x_i$}}{s,t} & \sdefine &
%%   \sema{\kind{qid}}{\semcode{$Xs$.$x_i$}}{s} \cand \CResolve{\ri{x}{i}}{\delta} \cand \CType{\delta}{t} & \fresh{\delta} \\
%% \sema{\kind{exp}}{\semcode{fun $\ f$ ( $x_i$:$t$ ) \{$e$\}}}{s,t} & \sdefine &
%%    \AParent{S'}{s} \cand 
%%    \AScopeof{\di{x}{i}}{S'}  \cand
%%    \CType{\di{x}{i}}{t'}  \cand \\
%%    && 
%%    C \cand
%%    \sema{\kind{exp}}{\semcode{$e$}}{S',\tau_2} \cand
%%    \CTeq{t}{\arrty{t'}{\tau_2}} 
%%    \ \ \ \ \ \mbox{\it where } \sema{\kind{ty}}{t}{s} = (t',C)
%%     & \new{S'}\fresh{\tau_2} \\
%% \sema{\kind{exp}}{\semcode{$e_1\ e_2$}}{s,t} & \sdefine &
%%    \sema{\kind{exp}}{\semcode{$e_1$}}{s,{\arrtysub{\tau_1}{t}}} \cand \sema{\kind{exp}}{\semcode{$e_2$}}{s,\tau_2} \cand \CSubtype{\tau_2}{\tau_1} &
%%    \fresh{\tau_1,\tau_2} \\
%% % sequential let omitted
%% \sema{\kind{exp}}{\semcode{letrec $\ bs\ $ in $\ e$}}{s,t} & \sdefine &
%%    \AParent{S'}{s} \cand
%%    \sema{\kind{defbind}^*}{\semcode{$bs$}}{S',S'} \cand
%%    \sema{\kind{exp}}{\semcode{$e$}}{S',t} & \new{S'} \\
%% \sema{\kind{exp}}{\semcode{letpar $\ bs\ $ in $\ e$}}{s,t} & \sdefine &
%%    \AParent{S'}{s} \cand
%%    \sema{\kind{defbind}^*}{\semcode{$bs$}}{S,S'} \cand
%%    \sema{\kind{exp}}{\semcode{$e$}}{S',t} & \new{S'} \\
%% \sema{\kind{exp}}{\semcode{new $\ Xs$.$X_i\ $\{ $bs$\}}}{s,t} & \sdefine &
%%    \sema{\kind{qid}}{\semcode{$Xs$.$X_i$}}{s} \cand 
%%    \CResolve{\ri{X}{i}}{\delta} \cand
%%    \CTeq{t}{\recty{\delta}} \cand \\
%%     & & \AParent{S'}{s} \cand
%%    \ARImport{\ri{X}{i}}{S'} \cand
%%    \sema{\kind{fldbind}^*}{bs}{s,s'}    
%%    & \new{S'}\fresh{\delta} \\
%% \sema{\kind{exp}}{\semcode{$e$.$x_i$}}{s,t} & \sdefine &
%%    \sema{\kind{exp}}{\semcode{$e$}}{s,\rectysub{\delta_1}} \cand
%%    \CAssoc{\delta_1}{\sigma} \cand
%%    \AParent{S'}{\undef} \cand
%%    \ASImport{\sigma}{S'} \cand \\
%%    & & \AScopeof{\ri{x}{i}}{S'} \cand 
%%    \CResolve{\ri{x}{i}}{\delta_2} \cand
%%    \CType{\delta_2}{t}
%%    & \new{S'}\fresh{\delta_1,\delta_2,\sigma}\\
%% \sema{\kind{exp}}{\semcode{with $\ e_1\ $ do $\ e_2$}}{s,t} & \sdefine &
%%    \sema{\kind{exp}}{\semcode{$e_1$}}{s,\rectysub{\delta}} \cand
%%    \CAssoc{\delta}{\sigma} \cand 
%%    \AParent{S'}{s} \cand
%%    \ASImport{\sigma}{S'} \cand 
%%    \sema{\kind{exp}}{\semcode{$e_2$}}{S',t} 
%%    & \new{S'}\fresh{\delta,\sigma} \\[0.3cm]
%% %\\
%% %\\
%% %%\sema{fldbinds}{\semcode{$b\ bs$}}{s_r,s_d} & \sdefine &
%% %%   \sema{fldbind}{\semcode{$b$}}{s_r,s_d} \cand \sema{fldbinds}{\semcode{$bs$}}{s_r,s_d}\\
%% %%\sema{fldbinds}{}{s_r,s_d} & \sdefine & \CTrue \\
%% %%\\
%% \sema{\kind{fldbind}}{\semcode{$x_i\ $=$\ e$}}{s_r,s_d} & \sdefine &
%%    \AScopeof{\ri{x}{i}}{s_r} \cand
%%    \CResolve{\ri{x}{i}}{\delta} \cand
%%    \CType{\delta}{\tau_1} \cand
%%    \CSubtype{\tau_2}{\tau_1} \cand
%%    \sema{\kind{exp}}{\semcode{$e$}}{s_r,\tau_2} 
%%    & \fresh{\delta,\tau_1,\tau_2} \\[0.3cm]
%% %\\
%% \sema{\kind{ty}}{\semcode{Int}}{s} & \sdefine &
%%   (\intty, \CTrue) \\
%% \sema{\kind{ty}}{\semcode{$t_1$ -> $t_2$}}{s} & \sdefine &
%%   (\arrty{t'_1}{t'_2}, \CAnd{C_1}{C_2}) \ \ \ \ \ \ \mbox{\it where } \sema{\kind{ty}}{t_1}{s} = (t'_1,C_1) \mbox{\it\ and } \sema{\kind{ty}}{t_2}{s} = (t'_2,C_2) \\
%% \sema{\kind{ty}}{\semcode{$Xs$.$X_i$}}{s} & \sdefine &
%%   (\recty{\delta}, \CAnd{\sema{\kind{qid}}{\semcode{$Xs$.$X_i$}}{s}}{\CResolve{\ri{X}{i}}{\delta}})  & \fresh{\delta} \\ 
%% %% APT: or, another way, that passes a type down and returns just a constraint instead of a tuple:
%% %%\sema{ty}{\semcode{Int}}{s,t} & \sdefine &
%% %%   \CTeq{t}{\intty} \\
%% %% \sema{ty}{\semcode{$t_1$ -> $t_2$}}{s,t} & \sdefine &
%% %%   \CTeq{t}{\arrty{t'_1}{t'_2}} \cand
%% %%   \sema{ty}{t_1}{s,t'_1} \cand
%% %%   \sema{ty}{t_2}{s,t'_2} & \fresh{t'_1,t'_2} \\
%% %% \sema{ty}{\semcode{$xs$.$x_i$}}{s,t} & \sdefine &
%% %%   \CTeq{t}{\recty{\delta}} \cand
%% %%   \sema{qid}{\semcode{$xs$.$x_i$}}{s} \cand
%% %%    \CResolve{\ri{x}{i}}{\delta}   & \fresh{\delta} \\ 
%% \end{array}
%% $
%%   \caption{Constraint generation for LMR.}
%%   \label{fig:consgen}
%% \end{figure*}


%% \newcommand{\cg}[2]{\vdash #1 \longrightarrow #2}
%% \newcommand{\cgi}[3]{\vdash #1_{#2} \longrightarrow #3}
%% %\newcommand{\f}[1]{fresh(#1)}
%% \newcommand{\pa}[2]{P(#1,#2)}
%% \renewcommand{\w}{\wedge}
%% \newcommand{\getx}[3]{#1 \stackrel{#2}{\longrightarrow} #3}
%% \newcommand{\DS}[1]{#1^D}
%% \newcommand{\TD}[1]{#1^T}

%% \begin{figure*}

%% \centering


%% \infrulenl{S \cg{sts}{c}}{\cg{Prog(sts)}{\pa{S}{\bot} \wedge c}}

%% \medskip

%% \infrulenl{S \cg{s}{c_1}\tab S \cg{sts}{c_2}}{\cg{s :: sts}{c_1 \w c_2}}

%% \medskip

%% \infrulenl{S' \cg{sts}{c}}{S \cg{Mod(x_i,sts)}{\dsi{x}{i}{S'} \in \D{S} \wedge \pa{S'}{S} \w c}}

%% \medskip

%% \infrulenl{\cg{qid}{c} \tab \getx{qid}{\textsf{R}}{\ri{id}{i}} }{S \cg{Imp(qid)}{\ri{id}{i} \in \I{S} \w c}}

%% \medskip

%% \infrulenl{\DS{S} \cg{b}{c}}{S \cg{Def(b)}{c}}

%% \medskip

%% \infrulenl{\DS{S} \cg{decls}{fc} \tab \DS{S'}\ \TD{\recty{\di{x}{i}}} \cg{sup}{sc} }
%% {S 
%%   \cg{Rec(x,sup,dec)_i}
%%   {\dsi{x}{i}{S'} \in \D{S} \w 
%%     \recty{\di{x}{i}} \Aassoc S' \w 
%%     \pa{S'}{S} \w 
%%     i : \recty{\di{x}{i}} \w
%%     fc \w 
%%     sc}}

%% \medskip

%% \infrulenl{\cg{qtid}{tc} \tab \getx{qtid}{\textsf{R}}{\ri{x}{i}}}
%% {\DS{S}\ \TD{T} 
%%   \cg{Super(qid)}{
%%     \ri{x}{i} \in \I{S} \w
%%     \ri{x}{i} \resolve \delta \w
%%     ty <: \recty{\delta} \w
%%     tc}}

%% \medskip

%% \infrulenl{}{\TD{T} \cg{NoSuper}{T <: \bot}} (is it necessary ?)

%% \medskip

%% \infrulenl{}{\TD{T} \cg{Int(n)}{i : T \w Int <: T}} (Why Ftypeof ?)

%% \medskip

%% \infrulenl{op :: T_1 * T_2 \rightarrow T3 \tab \TD{T_1} \cg{e_1}{c_1} \tab \TD{T_2} \cg{e_2}{c_2}}
%% {\TD{ty} \cg{Bop(op,e_1,e_2)_i}{i : ty \w T_3 \leq ty \w c_1 \w c_2 }}

%% \medskip

%% \infrulenl{\TD{\mathbb{B}} \cg{e_1}{c_1} \tab \TD{T} \cg{e_2}{c_2} \tab  \TD{T} \cg{e_3}{c_3}}
%% {\TD{T} \cg{Ift(e_1,e_2,e_3)_i}{i : T \w c_1 \w c_2 \w c_3}}
  
%% \medskip

%% \infrulenl{\cg{qid}{c} \tab \getx{qid}{\textsf{R}}{\ri{x}{i'}}}
%%    {\TD{T} \cg{Ref(qid)_i}{i : T \w \ri{x}{i'} \resolve \delta \w \pos{\delta} : \tau \w i' : \tau \w \tau \leq T \w c}}

%% \medskip

%% \infrulenl{\DS{S'}\ \TD{\tau} \cg{b}{c_b} \tab S'\ \TD{\tau} \cg{e}{c_e} }{S\ \TD{T} \cg{Fix(b,e)_i}{i : T \w \pa{S'}{S} \w \tau \leq T \w c_b \w c_e}} (This rule is wrong in DS)

%% \medskip

%% \infrulenl{\DS{S'}\ \TD{\tau_1} \cg{b}{c_b} \tab S'\ \TD{\tau_2} \cg{e}{c_e} }{S\ \TD{T} \cg{Fun(b,e)_i}{i : T \w \pa{S'}{S} \w \TArrow{\tau_1}{\tau_2} \leq T \w  c_b \w c_e}}

%% \medskip

%% \infrulenl{\TD{(\TArrow{\tau_1}{T})} \cg{e_1}{c_1} \tab \TD{\tau_1} \cg{e_2}{c_2} }{\TD{T} \cg{App(e_1,e_2)_i}{i : T \w  c_1 \w c_2}}


%%   \caption{Constraint generation}
%%   \label{fig:consgen}
%% \end{figure*}




%\lstinputlisting[language=DynSem,basicstyle=\tiny]%
%  {../../trunk/LMR/trans/extraction/extraction-equal.ds}
