\section{Constraint Language}\sectionlabel{constraintlang}

\newcommand{\sort}[1]{\text{\sc #1}}

In this section we introduce the syntax and declarative semantics of
constraints.

\subsection{Syntax of Contraints}

\Figure{constraintsyntax} defines the syntax of constraints.
The language independent base terms of the constraint language are:
\begin{itemize}
 \item \emph{Declarations} in $\sort{D}$, which are either ground declarations
 $\di{x}{i}$ of the  program or variables $\delta$
 \item \emph{References} in $\sort{R}$, which are either ground references
 $\ri{x}{i}$ of the program  or variables $\rho$
 \item \emph{Scopes} in $\sort{S}$, which are either ground scopes of the
 program denoted $S_i$ or variables $\varsigma$
 \item \emph{Positions} in $\sort{I}$, which can be ground positions $i$,
 positions of a declaration  or a reference $Pos(d)/Pos(r)$ or variables $\iota$
 \item \emph{Types} in $\sort{T}$, which are either type variables $\tau$ or
 type constructor applications $c({\sort{T},...,\sort{T}})$ with $c \in
 C_{\mathcal{T}}$, the set of language-specific type constructors. 
\end{itemize}

% Therefore we assume a set of type constructor for the
% language $\TypeCons$, each constructor having a corresponding signature $c ::
% \kappa_1*\dots*\kappa_n$ of the sorts of the arguments it expects. Types in the
% sort $\sort{T}$ can be constructed with these type constructors and type
% variables $\tau$.

\noindent Given these terms we define the syntax of constraints, which come in
two flavors, \emph{assumptions} and pure \emph{constraints}.
\emph{Assumptions}, defined by the sort $\sort{A}$, correspond to known facts
about a program: 
the scope of a reference ($\AScopeof{\sort{R}}{\sort{S}}$),
the scope of a declaration ($\AScopeof{\sort{D}}{\sort{S}}$),
the associated scope of a declaration ($\AAssoc{\sort{D}}{\sort{S}}$),
the parent of a scope ($\AParent{\sort{S}}{\sort{S}}$),
a named import in a scope ($\ARImport{\sort{S}}{\sort{S}}$),
a direct import in a scope ($\ASImport{\sort{S}}{\sort{S}}$),
and a subtype relation between types ($\ASubtype{\sort{T}}{\sort{T}}$).
\emph{Constraints}, defined by the sort $\sort{C}$, represent the restrictions
on name and type resolution, which consist of:
resolution of a reference to a declaration ($\CResolve{\sort{R}}{\sort{D}}$),
equality of two types ($\CTeq{\sort{T}}{\sort{T}}$),
subtype relation between two types ($\CSubtype{\sort{T}}{\sort{T}}$),
associated scope of a declaration ($\CAssoc{\sort{D}}{\sort{S}}$),
the type of a declaration ($\CType{\sort{D}}{\sort{T}}$),
and the least upper bound of two sorts ($\CLub{\sort{T}}{\sort{T}}{\sort{T}}$).
Assumptions and constraints can be combined using conjunction 
($\CAnd{\sort{C}}{\sort{C}}$) and $\true$ represents
the absence of constraints.  As before, we use different colors to help
distinguish between \assumpcolor{assumptions on the scope graph}, 
\subtypecolor{assumptions on the subtyping relation}, 
\typconstraintcolor{pure typing constraints}, and 
\resconstraintcolor{pure resolution constraints}.

\paragraph{Language-Specific Types}

The language of constraints defined above is independent of the language under
analysis, except for the type constructors introduced by this language. 
Therefore we assume a set of language specific types constructors $C_{\mathcal{T}}$
and each constructors $c$ has an associated arity $c :: n$. 
For example $\intcon$ and $\boolcon$ are type constructor with arity $0$ and
$\arrcon$ has arity 2. 
To represent user-defined types, such as classes in object-oriented languages or algebraic
data types in functional languages, a type constructor can also include the identity 
of the type definition. This identity is represented by the corresponding 
declaration (in the formal sense of scope graphs).
For example, record types in LMR are represented by $Rec(d)$ with $d$ a
declaration in the program
Thus, in \Figure{field-access}, the record definition $A_1$ defines the type
$Rec(\di{A}{1})$



\begin{figure}[t]
\begin{boxedminipage}{\hsize}
  \begin{align*}
& \begin{array}{rlll}
  \sort{A} := & \AScopeof{\sort{R}}{\sort{S}} 
  & \mid \AAssoc{\sort{D}}{\sort{S}}  
  & \mid \ARImport{\sort{S}}{\sort{S}}\\ 
  \mid & \AScopeof{\sort{D}}{\sort{S}}
  & \mid \AParent{\sort{S}}{\sort{S}} 
  & \mid \ASImport{\sort{S}}{\sort{S}} \\
  \mid & \ASubtype{\sort{T}}{\sort{T}} \\
   % & \mid \AScopeDef{\sort{D}}{\sort{S}}\\
  \\
  \sort{C} := & \CTrue
  & \mid \CResolve{\sort{R}}{\sort{D}} 
  & \mid \CTeq{\sort{T}}{\sort{T}} \\
  \mid & \CAnd{\sort{C}}{\sort{C}} 
  & \mid \CAssoc{\sort{D}}{\sort{S}}
  & \mid \CSubtype{\sort{T}}{\sort{T}} \\
  \mid & \CType{\sort{D}}{\sort{T}} 
  & \mid \CLub{\sort{T}}{\sort{T}}{\sort{T}}
  & \mid \sort{A}
  % \mid  \exists\vec{\tau}\vec{\sigma}\vec{\iota}. \sort{C} \\
\end{array}\\
& \begin{array}{rlllrllll} 
\multicolumn{2}{l}{\sort{D} :=  \delta  \mid \di{x}{i}\tab \tab} 
& \multicolumn{6}{l}{\sort{S} := \varsigma  \mid S_i \mid \bot }
\smallskip\\
\multicolumn{2}{l}{\sort{R} := \rho \mid \ri{x}{i}} 
& \multicolumn{6}{l}{\sort{T} := \tau \mid
c({\sort{T},...,\sort{T}}) \hbox{ with } c \in \TypeCons}\smallskip\\
%\sort{I} :=  & \iota  & \mid i & \mid \pos{\sort{D}} & \mid \pos{\sort{R}} \smallskip\
\end{array}
\end{align*}
\end{boxedminipage}
  \caption{Syntax of constraints}
  \label{fig:constraintsyntax}
\end{figure}


\subsection{Semantics of Constraints}

In our approach, the abstract syntax tree of a program $p$ is reduced by a
language-specific extraction function $\semop{p}$ to a constraint following the
syntax defined in \Figure{constraintsyntax}. Given commutativity and
associativity of conjuction, such a constraint is equivalent to a constraint of
the form
\vspace*{-2mm}
$$\CAnd{A_1}{\CAnd{\ldots}{\CAnd{A_n}{\CAnd{C_1}{\CAnd{\ldots}{C_m}}}}}\vspace*{-2mm}$$

\noindent consisting of a set of assumptions $A_i$ and a set of constraints
$C_j$. (We define an example extraction function in the next section.) The
assumptions define the scope graph and subtyping relation with respect to which
the other constraints need to be solved.

\paragraph{Interpretation of Assumptions}\label{ssec:assumptionsem}

We denote with $A^{<:}$ the set of assumptions of the form $\ASubtype{T_1}{T_2}$
in $A$ that will be used to build the corresponding sub-typing relation. We
denote with $A^\G$ the subset formed by the other assumptions that will
define the scope graph of the program.
Given a ground set of assumption $A$, we denote $|A|$ the interpretation
of $A$ as the pair $\G,\leq$ where $\leq$ is the sub-typing relation derived
from $A^{<:}$ and $\G$ is the scope graph derived from $A^\G$.

% Note that unlike a constraint which does not have to be ground to be satisfied
% (the constraint $\CTeq{X}{X}$ is always satisfied), a set of assumptions has to
% be ground to be interpreted. This property of the constraints allows us to type
% the identity function as $X \rightarrow X$ where $X$ is a free type variable.

\paragraph{Sub-typing} 

From the set of assumptions $A^{<:}$ we derive the relation $\leq$ between
ground types, built using the type constructors in $\TypeCons$. A variance $v$
is a non-empty subset of $\{-,+\}$ written $-$ for contravariant, $+$ for
covariant and $\pm$ for invariant. We also denote $\leq$ for $\leq^{+}$, $\geq$
for $\leq^{-}$ and $=$ for $\leq^{\pm}$. We require all the arguments in the
signature of a constructor $c$ to be annotated with a variance parameter. Thus
the signature of a type constructor $c$ is declared as $c :: v_1*...* v_n$, with
the $v_i$ variance annotations. Given such a signature for all the type
constructors, we now define the sub-typing relation $\leq$ derived from a set
$A^{<:}$ of sub-typing assumptions by the following inductive rules:

\begin{center}$
\begin{array}{ccc}
 \infrulenl{}{T \leq T} & & \infrulenl{T_1 \leq T_2 \tab T_2 \leq T_3  }{T_1 \leq T_3}\bigskip\\
 \infrulenl{T_1 <: T_2 \in A^{<:}}{T_1 \leq T_2} & & 
 \infrulenl{c :: v_1*...*v_n\tab \forall i,\ s_i \leq^{v_i} t_i }{c(s_1,...,s_n)
 \leq c(t_1,...,t_n)} \end{array}$
\end{center}

% \TODO{notation for the subtyping relation associated with a set of constraints
% or a set of   <: constraints}

\paragraph{Scope graph} 

The assumptions that are not sub-typing assumptions define the scope graph of
the program. These assumptions define the set of scopes, declarations and
references and the corresponding relations as follows:

\begin{itemize}[leftmargin=2cm]
 \item[$\AParent{S}{S'}$] defines a new scope $S$ and declares its parent
 $\P{S}$ as $S'$ when  $S'$ is not $\bot$
 \item[$\AScopeof{\d{x}}{S}$] defines a new declaration $\d{x}$ and declares its
 enclosing scope $\scopeof(\d{x})$ as $S$
 \item[$\AAssoc{d}{S}$] declares scope $S$ as the associated scope
 $\defscope(d)$ of  declaration $d$
 \item[$\AScopeof{\r{x}}{S}$] defines a new reference $\r{x}$ and declares its
 enclosing  scope $\scopeof(\r{x})$ as $S$
 \item[$\ARImport{r}{S}$] adds the reference $r$ to the set of named imports
 $\I{S}$ of  scope $S$
 \item[$\ASImport{S'}{S}$] adds the scope $S'$ to the set of direct imports
 $\I{S}$ of  scope $S$
\end{itemize}

\noindent
The result is a correct scope graph according to \Section{extscopegraph}
provided that the parent relation is well-founded.


%%%%%%%%%%%%%%%
%% Next is the version with explicit subsitution \phi in context
%%%%%%%%%%%%%%%
% $\begin{array}{cl}
%   \infrulenl{}{\extenv,\psi,\phi \models \mlit{true}}
%   & \labl{C-True}\\
%   \infrulenl{\extenv,\psi,\phi \models C_1 \tab \extenv,\psi,\phi \models C_2}
%   {\extenv,\psi,\phi \models C_1 \wedge C_2}
%   & \labl{C-And}\\
%   \left( \infrulenl{\extenv,\psi,\phi[\vec\tau \mapsto \vec{t}, \vec\sigma \mapsto \vec{s}, \vec\iota \mapsto \vec{i}] \models C}
%     {\extenv,\psi,\phi \models \exists{\vec\tau \vec\sigma \vec\iota}.C}\right)
%   & \labl{C-Exists} \\
%   \infrulenl{\psi(\phi(i)) = \phi(T)}
%   {\extenv,\psi,\phi \models i : T }
%   & \labl{C-TypeOf} \\
%   \infrulenl{\Sigma(\phi(T)) = \phi(S)}
%   {\extenv,\psi,\phi \models T \Aasso S}
%   & \labl{C-ScopeOf} \\
%   \infrulenl{\vdash_{\G} p : x^R_{i_1} \mapsto x^D_{\phi(i_2)}}
%   {\extenv,\psi,\phi \models i_1 \mapsto i_2}
%   & \labl{C-Resolve} \\
%   \infrulenl{\phi(T_1) = \phi(T_2)}
%   {\extenv,\psi,\phi \models T_1 = T_2}
%   & \labl{C-Eq} \\
%   \infrulenl{\phi(T_1) \preceq \phi(T_2)}
%   {\extenv,\psi,\phi \models T_1 \leq T_2}
%   & \labl{C-SubType} \\
%   % \infrulenl{\vdash_{\G} p : \phi(S) \mapsto x^D_{\phi(I)} & \mathbf{P} \not\in p}
%   % {\extenv,\psi,\phi \models S \xmapsto{X} I}
%   % & \labl{C-ResolveIn} \\
% \end{array}$



\paragraph{Interpretation of Constraints}

\newcommand{\labl}[1]{\tab \text{\sc (#1)}\medskip}
\newcommand{\extenv}{\G,\leq}

The interpretation of a non-assumption constraints (which we will just call
constraints) is defined as a truth value in a context, which is a triple of the
following elements:
\begin{itemize}
 \item A scope graph $\G$, as defined in Section \ref{sec:extscopegraph}
 \item A sub-typing relation $\leq$ on ground types
 \item A typing environment $\psi$ mapping declarations in $\DG$ to types in 
 $\sort{T}$
\end{itemize}

\noindent A context $\extenv,\psi$ satisfies a constraint $C$ if the predicate
$\extenv,\psi \models C$ holds. This predicate is defined by the set of
inductive rules in \Figure{conssem}, where $=$ is the syntactic equality on
terms, $\vdash_{\G} r \resolve d$ the resolution relation in graph $\G$ and, when it exists, $\lubsem{\leq}{S}$ denotes the least upper bound of types
in $S$ according to order $\leq$.

\begin{figure}[t]
\begin{boxedminipage}{\hsize}
\centering
$\begin{array}{cl}
  \infrulenl{}{\extenv,\psi \models \CTrue}
  & \labl{C-True}\\
  \infrulenl{\extenv,\psi \models C_1 \tab \extenv,\psi \models C_2}
  {\extenv,\psi \models \CAnd{C_1}{C_2}}
  & \labl{C-And}\\
  % \left( \infrulenl{\extenv,\psi,\phi[\vec\tau \mapsto \vec{t}, \vec\sigma \mapsto \vec{s}, \vec\iota \mapsto \vec{i}] \models C}
  %   {\extenv,\psi \models \exists{\vec\tau \vec\sigma \vec\iota}.C}\right)
  % & \labl{C-Exists} \\
  \infrulenl{\psi(d) = T}
  {\extenv,\psi \models \CType{d}{T} }
  & \labl{C-TypeOf} \\
  \infrulenl{\vdash_{\G} p : r \mapsto d}
  {\extenv,\psi \models \CResolve{r}{d}}
  & \labl{C-Resolve} \\
  \infrulenl{\defscope(d) = S}
  {\extenv,\psi \models \CAssoc{d}{S}}
  & \labl{C-ScopeOf} \\
  \infrulenl{t_1 = t_2}
  {\extenv,\psi \models \CTeq{t_1}{t_2}}
  & \labl{C-Eq} \\
  \infrulenl{T_1 \leq T_2}
  {\extenv,\psi \models \CSubtype{T_1}{T_2}}
  & \labl{C-SubType} \\
  \infrulenl{T = \lubsem{\leq}{\{T_1,T_2\}}}
  {\extenv,\psi \models \CLub{T}{T_1}{T_2}}
  & \labl{C-Lub} \\
  % \infrulenl{\vdash_{\G} p : \phi(S) \mapsto x^D_{\phi(I)} & \mathbf{P} \not\in p}
  % {\extenv,\psi \models S \xmapsto{X} I}
  % & \labl{-ResolveIn} \\
\end{array}$

\end{boxedminipage}
  \caption{Interpretation of pure constraints}
  \label{fig:conssem}
\end{figure}



\subsection{Program Resolution}\sectionlabel{progresolutionspec} 

The goal of the resolution of the program $p$ is to build a multi-sorted
substitution $\phi$ and a typing environment $\psi$ such that, if $\semop{p} = \CAnd{A_p}{C_p}$ then the following property holds:\vspace*{-2mm}
\begin{equation}\tag{$\diamond$}
  \label{eq:resoleq}
 |\phi(A_p)|,\psi \models \phi(C_p)  \vspace*{-2mm}
\end{equation}
% \begin{equation}\tag{$\diamond$}
%   \label{eq:resoleq}
%   \frac{
%     \semop{p} = \CAnd{A_p}{C_p}, \quad \semop{A_p} = \extenv
%   }{
%      \phi(\extenv), \psi \models \phi(C_p)  
%   }
% \end{equation}
Where $\phi(E)$ denotes the application of the substitution $\phi$ to all the
variables appearing in $E$ that are in the domain of $\phi$.
Note that $\phi$ has to make $\phi(A_p)$ a set of ground assumptions in order to
be able to interpret it whereas some free variable may remain in $\phi(C_p)$.
When the proposition \ref{eq:resoleq} holds we say that $\psi$ and $\phi$
resolve $p$.


\endinput

\subsection{Program resolution}\sectionlabel{progresolutionspec} 

The collection of a program constraints generate a constraint that mixes
assumptions and constraints, moreover this general constraint contains variable,
including in its assumption sub-terms (see constraint generation for field
access in \TODO{pointer here}).
Thus the first step for the resolution of a program is to separate the
assumptions from the constraint to turn the generated constraint into a pure
constraint and build the set of assumption separately. This can be done using
the following rewrite rules:
$$ (A, \CAnd{a}{c}) \rightarrow (\{a\}\cup A,c) \tab\tab (A, \CAnd{c}{a})
\rightarrow (\{a\}\cup A,c)$$ where $a$ is an assumption.
Then given a program $p$ and $C$ the generated constraint for $p$, we denote
$A_p$ be set of assumption and $C_p$ the pure constraint such that $$
(\emptyset,C) \rightarrow (A_p,C_p)$$.
Both $A_p$ and $C_p$ contain free variables of different sorts, moreover, some
of these variables are even shared between constraints and assumptions.
The goal of the resolution of the program $p$ is to build a multi-sorted
substitution $\phi$ and a typing environment $\psi$ such that:
\begin{equation}\tag{$\diamond$}
  \label{eq:resoleq}
 \semop{\phi(A_p)},\psi \models \phi(C_p)  
\end{equation}

Where $\phi(E)$ denote the application of the substitution $\phi$ to all the
variables appearing in $E$ that are in the domain of $\phi$.
Note that $\phi$ has to make $\phi(A_p)$ a set of ground assumptions in order to
be able to interpret it whereas some free variable may remain in $\phi(C_p)$.
When the proposition \ref{eq:resoleq} holds we say that $\psi$ and $\phi$
resolve $p$.

\PN{Next paragraph is dubious, there is probably sub-typing involved ? Or maybe not ?}

\paragraph{Principal solution}
Allowing free variables in a solution allows us to state what the principal resolution of a program might be, that is a solution 
that allow to recover all the other possible resolutions. A pair $\phi$, $\psi$ is said to be the principal resolution of a program $p$ if it satisfies the formula \ref{eq:resoleq} and for all $\phi'$, $\psi'$ resolving $p$ then $\phi'$ and $\psi'$ are instances of $\phi$ and $\psi$, i.e.
$$ \exists \phi''\ s.t.\ \phi' = \phi''\cdot\phi \wedge \forall i, \phi''(\psi(i)) = \psi'(i)$$

\PN{What does it means now in our new setting ? since this is only spec for a given set of constraints, we might still want to include subtyping}
